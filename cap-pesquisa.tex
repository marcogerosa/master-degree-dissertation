% -*- root: dissertacao.tex -*-
%%%%%%%%%%%%%%%%%%%%%%%%%%%%%%%%%%%%%%%%%%%%%%%%%%%%%%%%%%%%%%%%%%%%%%%
\setlength{\parindent}{0pt}
\setlength{\textheight}{22cm}
\setlength{\parskip}{0.2cm}

% Para aumentar o espa�amento entre as linhas
\linespread{1.2}
%%%%%%%%%%%%%%%%%%%%%%%%%%%%%%%%%%%%%%%%%%%%%%%%%%%%%%%%%%%%%%%%%%%%%%%

\chapter{Pesquisa}

Este trabalho trata-se de uma pesquisa descritiva e explorat�ria. Descritiva porque este tipo de pesquisa visa observar, analisar, registrar e correlacionar os fatos ou fen�menos sem manipul�-los \cite{TipoPesquisa:01} e explorat�ria porque s�o abstra�das pr�ticas recorrentes objetivando descever sua natureza \cite{ProcessoPesquisa:04} na forma de \textit{code smells} se utilizando do formato de padr�es. \\

\textit{TODO: Confirmar se faz sentido ser explorat�ria ou se faz mais sentido ser experimental.}


\section{Hip�teses}

A fazer. \\


\section{Processo de Pesquisa}

A fazer. \\


\subsection{Coleta de Dados}

A fazer. \\

\subsection{An�lise dos Dados}

A fazer. \\


\section{Escrita dos Code Smells}

A fazer. \\


