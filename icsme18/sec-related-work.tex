\section{Related Work}


A constante e rápida evolução de tecnologias existentes e a criação de novas tecnologias faz com que diversos temas, como manutenibilidade de sistemas, estejam também em constante alta. Muitos pesquisadores vêm pesquisando sobre a existência de maus cheiros de código específicos a uma dada tecnologia, por exemplo, arcabouços Java \cite{ORMSmells}, arquiteturas MVC~\cite{AnicheSmellsMVC:17, MvcSmells:16, FinavaroAniche2016}, CSS \cite{CSSCodeSmell}, e fórmulas em planilhas \cite{SpreadsheetsSmells:12}.

Pesquisas relacionadas a maus cheiros em aplicativos Android ainda são poucas. Umme et al. \cite{Mannan_Dig_Ahmed_Jensen_Abdullah_Almurshed} afirmam que, das principais conferências de manutenção de sistemas, dentre 2008 a 2015, apenas 10\% dos artigos consideraram em suas pesquisas projetos Android. Nenhuma outra plataforma móvel foi considerada. As conferências consideradas foram: ICSE, FSE, OOPSLA/SPLASH, ASE, ICSM/ICSME, MRS e ESEM.

Verloop \cite{MobileSmells:13} investigou a presença de maus cheiros de código tradicionais propostos por Fowler \cite{Refactoring:99} (Método Longo, Classes Grande, Lista de Parâmetros Longa, Inveja dos Dados e Código Morto) em aplicativos Android para determinar se esses maus cheiros ocorrem mais frequentemente em ``classes núcleo'', classes no projeto Android que precisam herdar de classes do Android \acs{SDK}, como por exemplo \textit{Activities}, \textit{Fragments} e \textit{Services}, comparando com classes ``não núcleo''. Para isso, ele fez uso de 4 ferramentas de detecção automática de maus cheiros: JDeodorant, Checkstyle, PMD e UCDetector.
O autor afirma que classes núcleos tendem a apresentar os maus cheiros: Classe Deus, Método Longo, Comandos \textit{Switch} e Checagem de Tipo pela sua natureza de muitas responsabilidades. As classes mais observadas com esses maus cheiros foram \textit{Activities}, que é o principal componente da camada de apresentação Android. O autor também conclui que o mau cheiro tradicional Longa Lista de Parâmetros é menos provável de aparecer em classes núcleo pois, nessas classes, a maioria dos métodos são sobrecargas de métodos da classe herdada proveniente do Android SDK, e como para se realizar uma sobrecarga de método é necessário seguir a assinatura do método original, este normalmente não é afetado por este mau cheiro. 

Gottschalk et al. \cite{RemovingEnergySmells:12} conduziram um estudo sobre formas de detectar e refatorar maus cheiros de código relacionados ao uso eficiente de energia. Os autores compilaram um catálogo com 6 maus cheiros de código extraídos de outros trabalhos, e trabalharam sob um trecho de código Android para exemplificar um deles, o \textit{Carregar Recurso Muito Cedo}, quando algum recurso é alocado muito antes de precisar ser utilizado. 

Reimann et al. \cite{30QualitySmells:14} correlacionam os conceitos de mau cheiro, qualidade e refatoração a fim de introduzir o termo mau cheiro de qualidade. Os autores compilaram um catálogo de 30 maus cheiros de qualidade para Android. Todo o catálogo pode ser encontrado online\footnote{http://www.modelrefactoring.org/smell\_catalog} e os mesmos também foram implementados no arcabouço Refactory \cite{Reimann}. Os requisitos de qualidade tratados por Reimann et al. \cite{30QualitySmells:14} são: centrados no usuário (estabilidade, tempo de inicio, conformidade com usuário, experiência do usuário e acessibilidade), consumo inteligente de recursos de hardware do dispositivo (eficiência no uso de energia, processamento e memória) e segurança.

Hetch \cite{HetchDetectingAntipatternsAndroidApps:15} utilizou a ferramenta de detecção de maus cheiros Páprika\footnote{https://github.com/geoffreyhecht/paprika} para identificar 8 maus cheiros, dos quais 4 são tradicionais: Classe \textit{Blob} \cite{AntiPatterns:98}, Canivete Suíço \cite{AntiPatterns:98}, Classe Complexa \cite{Refactoring:99} e Método Longo \cite{Refactoring:99} e 4 são Android: \textit{Internal Getter/Setter} \cite{30QualitySmells:14}, \textit{No Low Memory Resolver} \cite{30QualitySmells:14}, \textit{Member Ignoring Method} \cite{30QualitySmells:14} e \textit{Leaking Inner Class} \cite{30QualitySmells:14}. O autor buscou os maus cheiros em 15 aplicativos Android populares como Facebook, Skype e Twitter. 
O autor afirma que os maus cheiros tradicionais são tão frequentes em aplicativos Android como em não Android, com exceção do Canivete Suíço. Essa afirmação nos leva a entender que ele teria comparado a presença dos maus cheiros tradicionais em sistemas tradicionais com os mesmos maus cheiros em aplicativos Android, entretanto, não há informações de como o autor obteve a informação da presença de maus cheiros em projetos de sistemas tradicionais para compará-la com o resultado obtido em aplicativos Android.
