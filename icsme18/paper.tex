\documentclass[conference]{IEEEtran}

\usepackage{cite}
\usepackage{xparse}
\usepackage[pdftex]{graphicx}
\usepackage[T1]{fontenc}
\usepackage[utf8]{inputenc}

\usepackage{booktabs}
\usepackage[autostyle]{csquotes}
\usepackage{subcaption}
\usepackage{epigraph}
\usepackage{multirow}
\usepackage[pdftex]{hyperref}
\usepackage[strings]{underscore}
\usepackage[usenames,svgnames,dvipsnames]{xcolor}

\newlength{\imagewidth}

\graphicspath{{../dissertation-v2/figures/}}

\newcommand{\acs}[1]{\textit{#1}}
\newcommand{\acl}[1]{\textit{#1}}

\NewDocumentCommand{\rot}{O{45} O{1em} m}{\makebox[#2][l]{\rotatebox{#1}{#3}}}%

\usepackage{mdframed}
% Barra lateral a citação direta
\newmdenv[
  leftmargin        = 1.5em,
  innerleftmargin   = 1em,
  innertopmargin    = 0pt,
  innerbottommargin = 0pt,
  innerrightmargin  = 0pt,
  rightmargin       = 0pt,
  linewidth         = 2pt,
  topline           = false,
  rightline         = false,
  bottomline        = false
]{leftbar}

\definecolor{lightgray}{gray}{0.97}

\newmdenv[
  innerleftmargin   = 1em,
  innertopmargin    = 1em,
  innerbottommargin = 1em,
  innerrightmargin  = 1em,
  linecolor         = gray,
  backgroundcolor   = lightgray,
  rightmargin       = 0pt,
  linewidth         = 0.5pt
]{square}

\begin{document}

\title{An Empirical Catalogue of Code Smells for the Presentation Layer of Android Apps}

\author{\IEEEauthorblockN{1\textsuperscript{st} Given Name Surname}
\IEEEauthorblockA{\textit{dept. name of organization (of Aff.)} \\
\textit{name of organization (of Aff.)}\\
City, Country \\
email address}
\and
\IEEEauthorblockN{2\textsuperscript{nd} Given Name Surname}
\IEEEauthorblockA{\textit{dept. name of organization (of Aff.)} \\
\textit{name of organization (of Aff.)}\\
City, Country \\
email address}
\and
\IEEEauthorblockN{3\textsuperscript{rd} Given Name Surname}
\IEEEauthorblockA{\textit{dept. name of organization (of Aff.)} \\
\textit{name of organization (of Aff.)}\\
City, Country \\
email address}
}

\maketitle

\begin{abstract} 
A task that constantly software developers need to do is
identify problematic code snippets so they can refactor, with the ultimate
goal to have constantly a base code easy to be maintained and evolved. For
this, developers often make use of code smells detection strategies. Although
there are many code smells cataloged, such as \textit{God Class}, \textit{Long
Method}, among others, they do not take into account the nature of the
project. However, Android projects have relevant features and untested to
date, for example the \texttt{res} directory that stores all resources used in
the application or an \textsc{Activity} by nature, accumulates various
responsibilities. Research in this direction, specific on Android projects,
are still in their infancy. In this dissertation we intend to identify,
validate and document code smells of Android regarding the presentation layer,
where major distinctions when compared to traditional designs. In other works
on Android code smells, were identified code smells related to security,
intelligent consumption of device's resources or somehow influenced the
experience or user expectation. Unlike them, our proposal is to catalog
Android code smells which influence the quality of the code. It developers
will have another ally tool for quality production code. 
\end{abstract}

\begin{IEEEkeywords}
component, formatting, style, styling, insert
\end{IEEEkeywords}

\input sec-intro
\input sec-background
\input sec-methodology
\input sec-results
\input sec-related-work
\input sec-threats
\input sec-conclusion

\bibliographystyle{IEEEtran}
\bibliography{IEEEabrv,refs}



\end{document}
