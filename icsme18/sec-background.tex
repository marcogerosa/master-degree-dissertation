\section{Background: Android's Presentation Layer}

São muitos os componentes e recursos disponibilizados pelo Android \acs{SDK}. Nossa pesquisa tem foco em analisar os elementos relacionados à camada de apresentação. Deste modo, para delimitarmos quais elementos Android seriam investigados, em termos de \textbf{\small componentes da camada de apresentação Android} (códigos Java que derivam do Android SDK) fizemos uma extensa revisão na documentação oficial do Android \cite{AndroidDeveloperSite2016} e chegamos em quatro: \textit{Activities}, \textit{Fragments}, \textit{Adapters} e \textit{Listeners}.

Em termos de \textbf{\small recursos Android}, todos são por natureza relacionados à camada de apresentação \cite{AndroidFundamentals}. O Android provê mais de quinze diferentes tipos de recursos \cite{AndroidResourceType}. Com o objetivo de focar a pesquisa, optamos por selecionar os recursos principais. Para isso, nos baseamos nos recursos existentes no projeto criado a partir do modelo padrão pelo Android Studio \cite{FirstApp2017}\footnote{Até a versão 3.0 do Android Studio, versão mais atual no momento desta escrita, o modelo de projeto padrão, que é pré-selecionado na criação de um novo projeto Android, é o \textit{Empty Activity}.}, são eles: recursos de \textit{Layout}, recursos de \textit{Strings}, recursos de \textit{Style} e recursos \textit{Drawable}.

A seguir introduzimos brevemente cada um dos 8 elementos da camada de apresentação Android, investigados nesta pesquisa:

\begin{enumerate}
  \item \textbf{\textit{Activity}} é um dos principais componentes de aplicativos Android e representa uma tela pelo qual o usuário pode interagir com a \acs{UI}. Possui um ciclo de vida, como mencionado na seção anterior. Toda \textit{Activity} deve indicar no método de retorno \textit{onCreate} o recurso de \textit{layout} que deve ser usado para a construção de sua \acs{UI} \cite{AndroidActivities2016, AndroidDevActivityReference}.

  \item \textbf{\textit{Fragments}} representam parte de uma \textit{Activity} e também devem indicar seu recurso de \textit{layout} correspondente. \textit{Fragments} só podem ser usados dentro de \textit{Activities}. Podemos pensar neles como ``sub-Activities''. \textit{Fragments} possuem um ciclo de vida extenso, com mais de dez métodos de retorno. Seu ciclo de vida está diretamente ligado ao ciclo de vida da \textit{Activity} ao qual ele está contido. O principal uso de \textit{Fragments} é para o reaproveitamento de trechos de \acs{UI} e comportamento em diferentes \textit{Activities}~\cite{AndroidFragments}.

  \item \textbf{\textit{Adapters}} são utilizados para popular a \acs{UI} com coleções de dados, como por exemplo, uma lista de \textit{e-mails}, onde o \textit{layout} é o mesmo para cada item da lista mas o conteúdo é diferente~\cite{AndroidLayouts}.

  \item \textbf{\textit{Listeners}} são interfaces Java que representam eventos do usuário, por exemplo, o \textit{OnClickListener} captura o clique pelo usuário. Essas interfaces costumam ter apenas um método, onde é implementado o comportamento desejado para responder a interação do usuário~\cite{AndroidUIEvents}.

  \item \textbf{\textit{Recursos de Layout}} são XMLs utilizados para o desenvolvimento da estrutura da \acs{UI} dos componentes Android. O desenvolvimento é feito utilizando uma hierarquia de \textit{Views} e \textit{ViewGroups}. \textit{Views} são caixas de texto, botões, dentre outros. \textit{ViewGroups} são \textit{Views} especiais pois podem conter outras \textit{Views}. Cada \textit{ViewGroup} organiza suas \textit{Views} filhas de uma forma específica, por exemplo: horizontalmente, em tabela, posicionamento relativo, dentre outros. Esta hierarquia pode ser tão simples ou complexa quanto se precisar, mas quanto mais simples, melhor o desempenho~\cite{AndroidLayoutResources, AndroidLayouts}.

  \item \textbf{\textit{Recursos de String}} são XMLs utilizados para definir textos, conjunto de textos usados no aplicativo. As principais vantagens de se usar recursos de \textit{String} é o reaproveitamento dos textos em diferentes \acs{UI}s e a facilidade para internacionalizar~\cite{AndroidStringResources}.

  \item \textbf{\textit{Recursos de Style}} são XMLs utilizados para a definição de estilos a serem aplicados nos XMLs de \textit{layout}. As principais vantagens em se utilizar recursos \textit{Styles} é separar o código de estrutura da \acs{UI} do código que define sua aparência e forma, e também possibilitar a reutilização de estilos em diferentes \acs{UI}s~\cite{AndroidStyleResources}.

  \item \textbf{\textit{Recursos de Drawable}} são arquivos gráficos utilizados na \acs{UI}. Esses arquivos podem ser imagens tradicionais, \texttt{.png}, \texttt{.jpg} ou \texttt{.gif}, ou XMLs gráficos. A principal vantagem dos XMLs gráficos está no tamanho do arquivo que é comumente bem menor do que imagens tradicionais e, diferente das imagens tradicionais onde é recomendado que se tenha mais de uma versão da mesma em resoluções diferentes, para XMLs gráficos só é necessário uma versão~\cite{AndroidDrawableResources}.
\end{enumerate}

