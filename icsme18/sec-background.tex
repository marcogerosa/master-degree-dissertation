




\section{Background: Android's Presentation Layer}

Our research focuses on analyzing the elements related to the presentation layer of Android apps, according to the architecture defined by the Android \acs{SDK}. So to delimit which Android elements would be investigated,
we reviewed the official Android documentation \cite{AndroidDeveloperSite2016}, from which we identified: Activities, Fragments, Adapters, and Listeners.

Resources are also related to the presentation layer \cite{AndroidFundamentals}. Android provides more than fifteen different resource types \cite{AndroidResourceType}. To focus this research, we relied on the existing resources in the project created from the default template of Android Studio \cite{FirstApp2017} 
\footnote{Up to version 3.0 of Android Studio, the most current version at the time of this writing, the standard design template, which is pre-selected in the creation of a new Android project is an Empty Activity.}
, which are: Layout, Strings, Style, and Drawable resources.

Here, we briefly introduce each of the eight elements of the Android presentation layer that were considered in this study:

\begin{itemize}

    \item \textbf{\textit{Activity}} is one of the major components of Android applications and represents a screen. 

    \item \textbf{\textit{Fragments}} represent parts of an \textit{Activity} and should also indicate their corresponding \textit{layout} feature. Fragments can only be used within Activities. 

    \item \textbf{\textit{Adapters}} are used to populate the \acs{UI} with collections of data.

    \item \textbf{\textit{Listeners}} are Java interfaces that represent user events, e.g., the \textit{OnClickListener} event captures the user click. These interfaces usually have only one method, where the desired behavior is implemented.

    \item \textbf{\textit{Layout Resources}} are XML files used for the development of the \acs{UI} structure of Android components. The development is done using a hierarchy of \textit {Views} and \textit {ViewGroups}. \textit{Views} are text boxes, buttons, etc. \textit {ViewGroups} are a collection of \textit{Views} with a definition of how these \textit {Views} should be shown.

    \item \textbf{\textit{String Resources}} are XMLs used to define sets of texts to be used in the application for internationalization.

    \item \textbf{\textit{Style Resources}} are XMLs used to define styles to be applied in \textit {layout} XMLs. Their goal is to separate code related to structure from code related to appearance and shape.

    \item \textbf{\textit{Drawable Resources}} are graphical files used in the user interface. These files can be traditional images or graphic XMLs. 

\end{itemize}
