
\section{Introduction}

\emph{``We are aware that good code matters, because we have had to deal with the lack of it for a long time''}, argues Robert Martin \cite{CleanCode:08}. In fact, we are aware that good code matter. But how do we know whether the quality of a code is low? One way to answer this question is by searching for \emph{code smells} in the code. Code smells are certain structures in the source code that indicate the violation of fundamental design principles and negatively impact the quality of the project \cite{Refactoring:14}. Code smells assist developers in identifying problematic code snippets so that they can be improved and software quality increased \cite{Refactoring:99}.


Several code smells have been catalogued \cite{Refactoring:99,CleanCode:08,Refactoring:14,Webster:95}, e.g., Long Methods and God Classes. Many of these code smells have been defined based on traditional concepts and technologies, such as object orientation and the Java language, which emerged during the 1970s and 1990s. In this paper, we call them ``traditional code smells.'' However, in the last decade, many new technologies have emerged, such as Android, which have raised questions such as ``do traditional code smells apply to new technologies?'' Or ``are there code smells specific to new technologies and not yet cataloged?''. Such questions inspired the curiosity of several researchers who decided to investigate code smells in specific technologies, such as CSS \cite{CSSCodeSmell}, Javascript \cite{JavascriptSmells}, MVC systems \cite{AnicheSmellsMVC:17,MvcSmells:16}, and spreadsheets \cite{SpreadsheetsSmells:12}.

Android is a mobile platform launched in 2008 by Google, in partnership with several other companies \cite{OHAReleasesAndroidSDK:07}. In 2011 it became the world's leading mobile platform, and since then has been increasing its share of the market, having reached 86\% \cite{GlobalSmartphoneSales:09-17} in 2017.
The platform also caught the attention of researchers in the area of software quality. Some investigated the existence of traditional bad smells in Android applications \cite{Hecht:15,DomainMatters,MobileSmells:13}. Others have investigated the existence of Android-specific code smells related to efficiency (good use of features like memory and processing) and usability (software capability to be understood) \cite{RemovingEnergySmells:12, 30QualitySmells:14}. Other researchers have focused on understanding Android development features that set them apart from traditional software development \cite{Mantyla2013}.



In this paper, we investigate the existence of code smells related to the Android presentation layer. While other research has investigated code smells in terms of efficiency and usability, we have looked for those related to maintainability, which deals with the ease of software being modified or improved. 

Our data was obtained through two questionnaires and an online code experiment, in a total of 316 participants. The first exploratory questionnaire asked developers about best and bad practices they use in the day to day development of the Android presentation layer. We got 45 responses, from which we derived 20 Android code smells. The second was a confirmatory questionnaire, where we validated the perception with 201 Android developers about the frequency and importance of the 20 proposed code smells. Finally, we conducted a code experiment with 70 Android developers in order to validate whether classes affected by these smells were perceived as problematic.

% Our results show that 1) there exist code smells specific to the Android presentation layer, 2) developers are considered frequent and important to be mitigated, and 3) are perceived as problematic by Android developers when identified in source code. 

The main contributions of this paper are:

\begin{itemize}
  \item A catalogue with 20 new code smells related to eight components of the Android's presentation layer: \textit{Activities}, \textit{Fragments}, \textit{Adapters}, and \textit{Listeners} (Components), \textit{Layouts}, \textit{Styles}, \textit{String}, and \textit{Drawables} (resources).
  
  \item The perception of Android developers on the frequency and importance of each of the 20 proposed code smells.

  \item An statistical analysis on the developers' perceptions of seven out of the 20 code smells, where we confirmed that developers perceive classes affected by the proposed smells as problematic, when compared to clean classes.
\end{itemize}


