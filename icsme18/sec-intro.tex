
\section{Introduction}

\emph{``Estamos cientes de que um bom código importa, pois já tivemos que lidar com a falta dele por muito tempo''} argumenta Robert Martin \cite{CleanCode:08}. De fato, estamos cientes de que bons códigos importam. Mas como saber se a qualidade de um código está baixa? Uma das formas de responder essa pergunta é buscando por \emph{maus cheiros} no código. Maus cheiros são certas estruturas no código que indicam a violação de princípios fundamentais de \textit{design} e impactam negativamente a qualidade do projeto \cite[p.~258]{Refactoring:14}. Maus cheiros auxiliam desenvolvedores na identificação de trechos de códigos problemáticos, de forma que possam ser melhorados e a qualidade do software incrementada \cite{Refactoring:99}. Nesta dissertação, anomalias de código e maus cheiros são sinônimos.

Existem diversos maus cheiros catalogados \cite{Refactoring:99, CleanCode:08, Refactoring:14, Webster:95}, Método Longo e Classe Deus são dois exemplos. Muitos desses maus cheiros foram definidos baseados em conceitos e tecnologias tradicionais, como orientação a objetos e Java \cite{Java}, que surgiram durante as décadas de 70 a 90. Nesta dissertação, os denominamos de ``maus cheiros tradicionais''. Entretanto, na última década surgiram muitas novas tecnologias, como por exemplo o Android, que levantaram questões como: ``os maus cheiros tradicionais se aplicam às novas tecnologias?'' ou ``existem maus cheiros específicos às novas tecnologias ainda não catalogados?''. Questões como essas instigaram a curiosidade de diversos pesquisadores que decidiram investigar maus cheiros em tecnologias específicas, como por exemplo o CSS \cite{CSSCodeSmell}, o Javascript \cite{JavascriptSmells}, o arcabouço Spring MVC \cite{MvcSmells:16} e fórmulas de planilhas do Google \cite{SpreadsheetsSmells:12}.

Android é uma plataforma móvel que foi lançada em 2008 pela empresa Google em parceria com diversas outras empresas \cite{OHAReleasesAndroidSDK:07}. Em 2011 se tornou mundialmente a principal plataforma para dispositivos móveis e desde então vem aumentando sua fatia de mercado, tendo em 2017 alcançado 86\% \cite{GlobalSmartphoneSales:09-17}. 

O Android também chamou a atenção de pesquisadores da área de qualidade de software. Alguns investigaram a existência de maus cheiros tradicionais em aplicativos Android \cite{Hecht:15, DomainMatters, MobileSmells:13}. Outros investigaram a existência de maus cheiros específicos ao Android relacionados à eficiência (boa utilização de recursos como memória e processamento) e usabilidade (capacidade do software em ser compreendido) \cite{RemovingEnergySmells:12, 30QualitySmells:14}. Outros pesquisadores focaram em entender características do desenvolvimento Android que os diferenciam do desenvolvimento de software tradicional \cite{Mantyla2013}.

Dentre as descobertas realizadas pelos pesquisadores, notou-se que maus cheiros específicos ao Android são muito mais frequentes em aplicativos Android do que maus cheiros tradicionais \cite{Hecht:15}. Os componentes Android mais afetados por maus cheiros tradicionais fazem parte da camada de apresentação, como \textit{Activities} e \textit{Adapters} \cite{Hecht:15, Mantyla2013, MobileSmells:13}, e em alguns aplicativos Android, códigos relacionados à camada de apresentação são maioria, em termos de linhas de código (\acs{LoC}, do inglês \acl{LoC}) \cite{Mantyla2013}. Vale ressaltar que a camada de apresentação Android também é composta por arquivos XML, chamados de recursos da aplicação, que são usados para a construção da interface com o usuário (\acs{UI}, do inglês \acl{UI}) \cite{AndroidFundamentals}. Nenhuma das pesquisas mencionadas considerou esses arquivos em suas análises. 

Nesta dissertação, investigamos a existência de maus cheiros de código relacionados à camada de apresentação Android. Enquanto outras pesquisas investigaram maus cheiros em termos de eficiência e usabilidade, nós buscamos por maus cheiros relacionados à manutenibilidade, que trata da facilidade do software de ser modificado ou aprimorado. Complementamos as pesquisas anteriores pois focamos na camada de apresentação Android considerando inclusive os recursos da aplicação. 

Nossos dados foram obtidos por meio de dois questionários e um experimento de código online. O primeiro questionário foi de cunho exploratório onde perguntamos a desenvolvedores sobre boas e más práticas utilizadas no dia a dia do desenvolvimento da camada de apresentação Android. Obtivemos 45 respostas, das quais derivamos 20 maus cheiros Android. O segundo, foi um questionário confirmatório, onde validamos a percepção com 201 desenvolvedores Android sobre a frequência e importância dos 20 maus cheiros propostos. Por último, realizamos um experimento de código com 70 desenvolvedores Android com o objetivo de validar se códigos afetados pelos maus cheiros eram percebidos como códigos problemáticos. Ao todo, participaram da pesquisa 316 desenvolvedores.

Nossos resultados mostram que, existem maus cheiros específicos à camada de apresentação Android, que eles são considerados frequentes e importantes de se mitigar e que são percebidos como problemáticos por desenvolvedores Android quando identificados em códigos. Nossas principais contribuições são: (i) um catálogo com 20 novos maus cheiros relacionados à camada de apresentação Android, (ii) uma análise estatística da percepção de desenvolvedores sobre 7 dos 20 maus cheiros catalogados e (iii) um apêndice \textit{online} \cite{apendice} com as informações necessárias para outros pesquisadores replicarem nossa pesquisa. 

Acreditamos que nossas contribuições dão um pequeno mas importante passo na busca por qualidade de código na plataforma Android e que poderá servir a pesquisadores e desenvolvedores. Aos pesquisadores serve como ponto de partida para a definição de heurísticas de identificação dos maus cheiros e implementação de ferramentas que os identifiquem de forma automática. Aos desenvolvedores serve como auxílio na identificação de códigos problemáticos para serem melhorados, ainda que de forma manual.

As principais contribuições desta dissertação são:

\begin{itemize}
  \item Um catálogo com 20 novos maus cheiros relacionados a 8 elementos da camada de apresentação Android, sendo quatro componentes: \textit{Activities}, \textit{Fragments}, \textit{Adapters} e \textit{Listeners} e quatro recursos: \textit{Layouts}, \textit{Styles}, \textit{String} e \textit{Drawables}.
  
  \item A validação da percepção de frequência e importância dos 20 maus cheiros no dia a dia de desenvolvimento Android.

  \item Uma análise estatística sobre a percepção de 7 dos 20 maus cheiros propostos por desenvolvedores Android onde foi possível confirmar que códigos afetados por 6 maus cheiros são percebidos como problemáticos por desenvolvedores Android.

  \item Um apêndice online \cite{apendice} com todos os relatórios e dados produzidos durante a pesquisa.
\end{itemize}


