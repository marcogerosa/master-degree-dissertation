
\section{Introduction}

\emph{``We are aware that good code matters, because we have had to deal with the lack of it for a long time''}, argues Robert Martin \cite{CleanCode:08}. However, how to know whether the quality of a piece of code is low? One way to answer this question is by searching for \emph{code smells} in the source code. Code smells are certain anomalies that may indicate the violation of design principles and negatively impact the quality of the project \cite{Refactoring:14}. While looking for code smells, developers identify problematic code snippets, creating opportunities for improving software quality \cite{Refactoring:99}.

Several code smells have been catalogued in the literature \cite{Refactoring:99,CleanCode:08,Refactoring:14,Webster:95}, e.g., Long Methods and God Classes. Many of these code smells have been defined based on traditional concepts and technologies, such as object orientation and the Java language, which emerged during the 1970s and 1990s. In this paper, we call them ``traditional code smells.'' However, in the last decade, many new technologies have emerged, raising questions such as ``do traditional code smells apply to new technologies?'' Or ``are there code smells specific to new technologies?''. Such kind of questions inspired several researchers who investigated code smells in specific technologies, such as CSS \cite{CSSCodeSmell}, Javascript \cite{JavascriptSmells}, MVC systems \cite{AnicheSmellsMVC:17,MvcSmells:16}, and spreadsheets \cite{SpreadsheetsSmells:12}. 

Android \cite{OHAReleasesAndroidSDK:07}, a mobile platform launched in 2008 by Google, has also attracted the attention of researchers in the area of software quality. Some investigated the existence of traditional bad smells in Android applications \cite{Hecht:15,DomainMatters,MobileSmells:13}. Others have investigated the existence of Android-specific code smells related to efficiency (good use of features like memory and processing) and usability (software capability to be understood) \cite{RemovingEnergySmells:12, 30QualitySmells:14}. Other researchers have focused on understanding Android development features that set them apart from traditional software development \cite{Mantyla2013}. However, no study focused on the Android presentation layer, which follows specific concepts and models. In this paper, we investigate the existence of code smells related to this part of the Android architecture. Different than other studies that focused on efficiency and usability, we have looked for code smells related to maintainability. 

We collected data employing two questionnaires and an online experiment, involving a total of 316 participants. The first exploratory questionnaire asked developers about best and bad practices they notice in the development of the Android presentation layer. We got 45 responses, from which we derived 20 Android code smells. The second was a confirmatory questionnaire with 201 Android developers investigating the frequency and importance of the 20 proposed code smells. Finally, we conducted an online experiment with 70 Android developers to validate whether classes affected by these smells were perceived as problematic.

% Our results show that 1) there exist code smells specific to the Android presentation layer, 2) developers are considered frequent and important to be mitigated, and 3) are perceived as problematic by Android developers when identified in the source code. 

The main contributions of this paper are:

\begin{itemize}
  \item A catalogue with 20 new code smells related to eight components of the Android's presentation layer: \textit{Activities}, \textit{Fragments}, \textit{Adapters}, and \textit{Listeners} (Components), \textit{Layouts}, \textit{Styles}, \textit{String}, and \textit{Drawables} (resources).
  
  \item The perception of Android developers on the frequency and importance of each of the 20 proposed code smells.

  \item An statistical analysis on the developers' perceptions of seven out of the 20 code smells, where we confirmed that developers perceive classes affected by the proposed smells as problematic when compared to clean classes.
\end{itemize}

% These results are particularly relevant as Android became the world's leading mobile platform since 2011, and since then has been increasing its share of the market, having reached 86\% \cite{GlobalSmartphoneSales:09-17} in 2017. 


