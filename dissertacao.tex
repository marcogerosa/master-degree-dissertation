\documentclass[12pt,letterpaper,oneside]{book} 
%\documentclass[12pt,twoside,letterpaper]{book}
% oneside indica que nao � frente e verso

% ---------------------------------------------------------------------------- %
% Pacotes 
\usepackage[T1]{fontenc}
\usepackage[brazil]{babel}
\usepackage[latin1]{inputenc}
\usepackage[pdftex]{graphicx}           % usamos arquivos pdf/png como figuras
\usepackage{setspace}                   % espa�amento flex�vel
\usepackage{indentfirst}                % indenta��o do primeiro par�grafo
\usepackage{makeidx}                    % �ndice remissivo
\usepackage[nottoc]{tocbibind}          % acrescentamos a bibliografia/indice/conteudo no Table of Contents
\usepackage{courier}                    % usa o Adobe Courier no lugar de Computer Modern Typewriter
\usepackage{type1cm}                    % fontes realmente escal�veis
\usepackage{listings}                   % para formatar c�digo-fonte (ex. em Java)
\usepackage{titletoc}
\usepackage{booktabs}                   % para gera��o de tabelas
%\usepackage[bf,small,compact]{titlesec} % cabe�alhos dos t�tulos: menores e compactos
\usepackage[fixlanguage]{babelbib}
\usepackage[font=small,format=plain,labelfont=bf,up,textfont=it,up]{caption}
\usepackage[usenames,svgnames,dvipsnames]{xcolor}
\usepackage[a4paper,top=2.54cm,bottom=2.0cm,left=2.0cm,right=2.54cm]{geometry} % margens
\usepackage[pdftex,plainpages=false,pdfpagelabels,pagebackref,colorlinks=true,citecolor=black,linkcolor=black,urlcolor=black,filecolor=black,bookmarksopen=true]{hyperref} % links em preto
% \usepackage[pdftex,plainpages=false,pdfpagelabels,pagebackref,colorlinks=true,citecolor=DarkGreen,linkcolor=NavyBlue,urlcolor=DarkRed,filecolor=green,bookmarksopen=true]{hyperref} % links coloridos
\usepackage[all]{hypcap}                    % soluciona o problema com o hyperref e capitulos
\usepackage[square,sort,nonamebreak,comma]{natbib}  % cita��o bibliogr�fica alpha (alpha-ime.bst)

% By David
\usepackage{amsthm}
\usepackage{acronym} 
% \usepackage[portugues,ruled,vlined,linesnumbered]{algorithm2e/algorithm2e}
\usepackage{supertabular}

% ---------------------------------------------------------------------------- %
% Cabe�alhos similares ao TAOCP de Donald E. Knuth
\usepackage{fancyhdr}
\pagestyle{fancy}
\fancyhf{}
\renewcommand{\chaptermark}[1]{\markboth{\MakeUppercase{#1}}{}}
\renewcommand{\sectionmark}[1]{\markright{\MakeUppercase{#1}}{}}
\renewcommand{\headrulewidth}{0pt}

% ---------------------------------------------------------------------------- %
\graphicspath{{./figuras/}}             % caminho das figuras (recomend�vel)
\frenchspacing                          % arruma o espa�o: id est (i.e.) e exempli gratia (e.g.) 
\urlstyle{same}                         % URL com o mesmo estilo do texto e n�o mono-spaced
\makeindex                              % para o �ndice remissivo
\raggedbottom                           % para n�o permitir espa�os extra no texto
\fontsize{60}{62}\usefont{OT1}{cmr}{m}{n}{\selectfont}
\cleardoublepage
\normalsize

% ---------------------------------------------------------------------------- %
% Op��es de listing usados para o c�digo fonte
% Ref: http://en.wikibooks.org/wiki/LaTeX/Packages/Listings
\lstset{ %
language=Java,                  % choose the language of the code
basicstyle=\footnotesize,       % the size of the fonts that are used for the code
numbers=left,                   % where to put the line-numbers
numberstyle=\footnotesize,      % the size of the fonts that are used for the line-numbers
stepnumber=1,                   % the step between two line-numbers. If it's 1 each line will be numbered
numbersep=5pt,                  % how far the line-numbers are from the code
showspaces=false,               % show spaces adding particular underscores
showstringspaces=false,         % underline spaces within strings
showtabs=false,                 % show tabs within strings adding particular underscores
frame=single,	                % adds a frame around the code
framerule=0.6pt,
tabsize=2,	                    % sets default tabsize to 2 spaces
captionpos=b,                   % sets the caption-position to bottom
breaklines=true,                % sets automatic line breaking
breakatwhitespace=false,        % sets if automatic breaks should only happen at whitespace
escapeinside={\%*}{*)},         % if you want to add a comment within your code
backgroundcolor=\color[rgb]{1.0,1.0,1.0}, % choose the background color.
rulecolor=\color[rgb]{0.8,0.8,0.8},
extendedchars=true,
xleftmargin=10pt,
xrightmargin=10pt,
framexleftmargin=10pt,
framexrightmargin=10pt
}

\pagestyle{headings}
\markboth{}{}

% ---------------------------------------------------------------------------- %
% Dimens�es da p�gina (letterpaper)
%\setlength{\paperwidth}{216mm}
%\setlength{\topmargin}{1.3cm}         % deslocamento do topo do texto 
%\setlength\oddsidemargin{0cm}
%\setlength\evensidemargin{0cm}
%\setlength{\parskip}{1.2mm}
%\setlength{\parindent}{4mm}
%\setlength{\textwidth}{135mm}          % largura do texto
%\setlength{\parindent}{0pt}
%\setlength{\textheight}{22cm}
%\setlength{\parskip}{0.2cm}


\newcommand{\eb}{\varepsilon}
\newcommand{\mdp}{\langle\mathcal{S,A},p,r,c\rangle}
\newcommand{\ctlstar}{{\sc ctl}$^\star$}
\newcommand{\ctl}{\sc ctl}
\newcommand{\ltl}{\sc ltl}
\newtheorem{Def}{Defini��o}[chapter]
\newtheorem{Teo}{Teorema}[chapter]
\newtheorem{Ex}{Exemplo}[section]
\newtheorem{Tab}{Tabela}[chapter]

\begin{document}
%\hypersetup{
%pdfauthor = {Suelen Goularte Carvalho},
%pdftitle = {Algo Relacionado a Mobile},
%pdfsubject = {Disserta��o de Mestrado},
%pdfkeywords={Mobile, Computa��o Movel, Celular} % <== Precisa rever o que vai colocar aqui !!!
%pdfcreator = {LaTeX with hyperref package},
%}

\frontmatter

\onehalfspacing
% -*- root: dissertacao.tex -*-
\thispagestyle{empty}
\begin{center}
    \vspace*{2cm}
    \large{\textbf{Detec��o de Anomalias na Camada de Apresenta��o\\
    de Aplicativos Android Nativos}}\\
	
    \vspace*{1.2cm}
    Suelen Goularte Carvalho \\ 
    
    \vskip 2cm
	\textsc{
	Disserta��o apresentada\\[-0.25cm] 
	ao\\[-0.25cm]
	Instituto de Matem�tica e Estat�stica\\[-0.25cm]
	da\\[-0.25cm]
	Universidade de S�o Paulo\\[-0.25cm]
	para\\[-0.25cm]
	obten��o do t�tulo\\[-0.25cm]
	de\\[-0.25cm]
	Mestre em Ci�ncias}
    
    \vskip 1.5cm
    Programa: Mestrado em Ci�ncia da Computa��o\\
    Orientador: Marco Aur�lio Gerosa, Ph.D.\\
   
    % \vskip 1.5cm

    \vskip 1cm
	\normalsize{}
	
    \vskip 0.5cm
    \normalsize{S�o Paulo, Julho de 2016}

\end{center}

% P�gina de rosto
\newpage
\thispagestyle{empty}
	\begin{center}
    	\vspace*{0.2 cm}
        \large{\textbf{Detec��o de Anomalias na Camada de Apresenta��o\\
    	de Aplicativos Android Nativos}}\\
	    \vspace*{2 cm}
	\end{center}

	\vskip 2cm

	\begin{flushright}
	Esta � a vers�o original da disserta��o elaborada\\
	pela candidata Suelen Goularte Carvalho, tal como\\
	submetida a Comiss�o Julgadora.\\
	\vskip 3cm

	\end{flushright}
	\vskip 4.2cm

	\begin{quote}
	\noindent Comiss�o Julgadora:
	
	\begin{itemize}
		\item {Marco Aur�lio Gerosa, Ph.d. $-$ IME-USP}
		\item {Alfredo Goldman vel Lejbman, Ph.d. $-$ IME-USP}
		\item {Paulo Roberto Miranda Meirelles, Ph.d. $-$ IME-USP}
	\end{itemize}
	  
	\end{quote}

\newpage
\thispagestyle{empty}
	\vspace*{12cm}
	\vskip 1cm

	\begin{flushright}
	{\small Dedico esta disserta��o de mestrado a minha m�e.\\}
	\end{flushright}

	\vspace*{1cm}

	\begin{flushright}
	{\it ``O motivo do tempo � que tudo n�o acontece de uma vez s�.''} \\
	$-$ Albert Einstein
	\end{flushright}

\pagebreak


\pagenumbering{roman}

\onehalfspacing
\chapter*{Agradecimentos}
\setlength{\parindent}{0mm}

A fazer. \\

% -*- root: dissertation.tex -*-
\noindent CARVALHO, G. S. \textbf{Anomalias na Camada de Apresenta��o Android: Um Estudo Sobre a Percep��o dos Desenvolvedores}. 
2018. %100 f.
Disserta��o (Mestrado) - Instituto de Matem�tica e Estat�stica,
Universidade de S�o Paulo, S�o Paulo, 2018.
\\

N�o h� d�vidas de que bons c�digos importam, mas como saber quando a qualidade est� baixa? Maus cheiros de c�digo nos auxiliam na identifica��o de trechos de c�digo problem�ticos, por�m a maioria dos maus cheiros catalogados se baseiam em tecnologias tradicionais, criadas dentre as d�cadas de 70 a 90, como Java. Ainda h� d�vidas sobre maus cheiros em tecnologias que surgiram na �ltima d�cada, como o Android, principal plataforma m�vel em 2017 com mais de 86\% de participa��o de mercado. Alguns pesquisadores derivaram maus cheiros Android relacionados a efici�ncia e usabilidade. Outros notaram que maus cheiros espec�ficos ao Android s�o muito mais frequentes nos aplicativos do que maus cheiros tradicionais. Diversas pesquisas conclu�ram que os componentes Android mais afetados por maus cheiros tradicionais pertencem a camada de apresenta��o, conhecida por \textit{front-end}, como \textit{Activities} e \textit{Adapters}. Notou-se tamb�m que em alguns aplicativos, c�digos do \textit{front-end} representam a maior parte. Vale ressaltar que o \textit{front-end} Android tamb�m � composto por arquivos \texttt{XML}, chamados de recursos, usados na constru��o da interface do usu�rio (\textit{User Interface} - UI), por�m nenhuma das pesquisas citadas os considerou em suas an�lises. Nesta disserta��o, investigamos a exist�ncia de maus cheiros relacionados ao \textit{front-end} Android considerando inclusive os recursos. Fizemos isso por meio de dois question�rios online e tr�s experimentos totalizando a participa��o de 3XX desenvolvedores. Nossos resultados mostram a exist�ncia de uma percep��o comum entre desenvolvedores Android praticantes sobre m�s pr�ticas no desenvolvimento do \textit{front-end} Android. Nossas principais contribui��es s�o um cat�logo com 13 maus cheiros do \textit{front-end} Android e uma an�lise estat�stica sobre a percep��o de desenvolvedores sobre os maus cheiros catalogados. Nossas contribui��es servir�o a pesquisadores como ponto de partida para a defini��o de heur�sticas e implementa��o de ferramentas automatizadas e a desenvolvedores como aux�lio na identifica��o de c�digos problem�ticos para serem melhorados, ainda que de forma manual.

\noindent \textbf{Palavras-chave:} engenharia de software, android, maus cheiros, qualidade de c�digo, manuten��o de software, anomalias de software.


\onehalfspacing
\tableofcontents

\chapter{Lista de Abreviaturas}

\begin{acronym}

\acro{SDK}{{\it Software Development Kit}} % 
\acro{IDE}{{\it Integrated Development Environment}} % 
\acro{APK}{{\it Android Package}} % 
\acro{ART}{{\it Android RunTime}} % 

\end{acronym}


\chapter{Lista de S�mbolos}

\begin{supertabular}{ll}


$\Sigma$ & Sistema de transi��o de estados \\


\end{supertabular}


\listoffigures
% \listofalgorithms

\mainmatter

%%%%%%%%%%%%%%%%%%%%%%%%%%%%%%%%%%%%%%%%%%%%%%%%%%%%%%%%%%%%%%%%%%%%%%%%%
\onehalfspacing

% -*- root: dissertacao.tex -*-
%%%%%%%%%%%%%%%%%%%%%%%%%%%%%%%%%%%%%%%%%%%%%%%%%%%%%%%%%%%%%%%%%%%%%%%
\setlength{\parindent}{0pt}
\setlength{\textheight}{22cm}
\setlength{\parskip}{0.2cm}

% Para aumentar o espa�amento entre as linhas
\linespread{1.2}
%%%%%%%%%%%%%%%%%%%%%%%%%%%%%%%%%%%%%%%%%%%%%%%%%%%%%%%%%%%%%%%%%%%%%%%

\chapter{Introdu��o}


\textit{``Qualquer tolo consegue escrever c�digo que um computador consegue entender. Bons programadores conseguem escrever c�digo que humanos conseguem entender''} \cite{1999:RID:311424} � uma cita��o popular conhecida por desenvolvedores. De fato, escrever c�digo de f�cil manuten��o pode ser desafiador. Por este motivo � comum desenvolvedores seguirem boas pr�ticas de programa��o j� estabelecidas \cite{FinavaroAniche2016}. Apesar de estas boas pr�ticas ajudarem a mitigar o decl�nio da quaidade de c�digo, na pr�tica � dif�cil escrever \textit{``c�digo limpo''}. 


Tal como qualquer projeto de software, ao se implementar um aplicativo Android, buscamos escrever um c�digo de qualidade e de f�cil manuten��o. De fato esta � uma tarefa desafiadora em qualquer projeto e por isso diversos estudos e livros j� foram publicados sobre esta tem�tica. � comum se aplicar boas pr�ticas de software consolidadas para projetos Java em projetos Android, devido a ambos se utilizarem da mesma linguagem e parad�gmas. No entanto, projetos Android trazem um desafio a mais, e diferentemente de outros projetos de software eles possuem algumas particularidades com rela��o a camada de apresenta��o. A estrutura de projetos Android se resume a um dir�torio \textit{java} onde ficam as classes java, um arquivo de configura��o chamado \textit{AndroidManifest.xml} que cont�m configura��es tais como primeira tela, permiss�es dentre outros e um diret�rio chamado \textit{res}, abrevia��o para o termo \textit{resources}, que cont�m subpastas com nomes pr�-determinados pelo Android, onde cada subpasta cont�m arquivos que desempenham pap�is espec�ficos na aplica��o. Por exemplo a subpasta \textit{layout} cont�m arquivos que representam a parte visual de telas do Android, a subpasta \textit{drawable} cont�m os elementos gr�ficos, por exemplo, imagens. J� a sub-pasta 


\section{Motiva��o}\index{motiva��o}

Smartphones existem desde 1993 por�m nesta �poca o foco era voltado para empresas e suas necessidades. Em 2007 houve o lan�amento do iPhone fabricado pela Apple, o primeiro smartphone voltado ao p�blico em geral. Ao final do mesmo ano o Google revelou seu sistema operacional para dispositivos m�veis, o Android \cite{SmartSociety:13}.

Nos anos que se seguiram diversas grandes empresas como Apple, Google, Blackberry e Microsoft entraram no mercado de smartphones e sistemas operacionais m�veis. Desde ent�o a ado��o do uso de smartphones e mais recentemente outros dispositivos m�veis conhecidos como dispositivos m�veis vest�veis (do ingl�s \textit{wearables}) tais como rel�gios inteligentes, pulseiras, dentre outros, tem crescido espantosamente.

Dentro do mercado de smartphone o Android tem ganhado for�a, o que resulta em mais pessoas usando dispositivos rodando aplica��es android. Desta forma, a demanda por aplicativos tem aumentado de forma acentuada. Logo, a quantidade de aplicativos desenvolvidos tem aumentado em propor��es similares. Isso significa que mais e mais projetos de aplicativos Android vem sendo constru�dos. 

Este crescimento de projetos de aplicativos na plataforma m�vel Android despertou a necessidade e interesse em explorar em mais profundidade quais s�o as particularidades de desenvolvimento nesta plataforma. Este trabalho vem contribuir com uma cole��o de \textit{code smells} validados, particulares a plataforma Android. 
 \\


\section{Objetivos}

A fazer. \\


\section{Abordagem de Solu��o}\index{abordagem}

A fazer. \\


\section{Originalidade e Relev�ncia}\index{originalidade}

H� muitos anos pesquisas em torno de maus cheiros de c�digo s�o realizadas. J� existem inclusive diversos maus cheiros mapeados, por�m poucos deles s�o feitos com rela��o a aplicativos Android nativos. Segundo artigos recentes [][] o estudo sobre maus cheiros de c�digo sobre aplicativos Android ainda est�o em sua inf�ncia, por�m ainda assim s�o muito relevantes inclusive notando-se que � mais comum identificar em aplicativos Android maus cheiros mapeados exclusivamente para esta plataforma do que os maus cheiros tradicionais. \\

Apesar de alguns esfor�os j� realizados neste sentido, a maior parte dos maus cheiros identificados espec�ficamente sobre a plataforma Android s�o espec�ficos sobre desempenho, economia de energia ou alguma outra caracter�stica que beneficie de alguma forma a experi�ncia do usu�rio, e n�o a qualidade e manuten��o de c�digo.


\section{Organiza��o do Trabalho}\index{organiza��o}

A fazer. \\

 
%%%%%%%%%%%%%%%%%%%%%%%%%%%%%%%%%%%%%%%%%%%%%%%%%%%%%%%%%%%%%%%%%%%%%%%
\setlength{\parindent}{0pt}
\setlength{\textheight}{22cm}
\setlength{\parskip}{0.2cm}

% Para aumentar o espa�amento entre as linhas
\linespread{1.2}
%%%%%%%%%%%%%%%%%%%%%%%%%%%%%%%%%%%%%%%%%%%%%%%%%%%%%%%%%%%%%%%%%%%%%%%

\chapter{Fundamenta��o Te�rica}

Planejamento\index{planejamento} � um componente importante do comportamento racional pois trata-se de um processo de s�ntese que seleciona e organiza a��es antecipando seus efeitos. Este processo busca satisfazer, da melhor forma poss�vel, um conjunto de metas pr�-definidas. \\

Muitas tarefas\index{tarefa} humanas necessitam de planejamento. Lorem ipsum dolor sit amet, consectetur adipiscing elit, sed do eiusmod tempor incididunt ut labore et dolore magna aliqua. Ut enim ad minim veniam, quis nostrud exercitation ullamco laboris nisi ut aliquip ex ea commodo consequat. Duis aute irure dolor in reprehenderit in voluptate velit esse cillum dolore eu fugiat nulla pariatur. Excepteur sint occaecat cupidatat non proident, sunt in culpa qui officia deserunt mollit anim id est laborum. \cite{RomanWeerdt2004}. Lorem ipsum dolor sit amet, consectetur adipiscing elit, sed do eiusmod tempor incididunt ut labore et dolore magna aliqua. O Exemplo \ref{ex:viagem_curitiba} ilustra um problema real de planejamento. \\


\section{Model View Controller}

Lorem ipsum dolor sit amet, consectetur adipiscing elit. Cras sodales turpis dolor, in porta justo sollicitudin eget. Duis sodales scelerisque viverra. Donec vulputate quam non diam ultricies, nec mattis mauris aliquam. Donec dictum dui ac dictum bibendum. Nunc placerat lobortis euismod. In pellentesque lectus imperdiet eros elementum, et viverra sapien fermentum. Aenean ultricies eu nulla eu molestie. Vivamus in vestibulum eros, rhoncus laoreet magna. Etiam imperdiet vestibulum metus, ac eleifend augue ornare cursus. Pellentesque vitae est purus. Ut eget ex libero. Mauris et sem iaculis, aliquet quam eget, feugiat nisl. Etiam porttitor sollicitudin facilisis. Donec egestas, diam sed tristique vestibulum, leo dolor cursus dui, non placerat lacus erat a augue. Nam ac consequat lectus.

Pellentesque rhoncus lacinia varius. Aliquam consectetur bibendum risus, in egestas tellus. Integer faucibus erat dui, ac interdum risus sodales in. Aenean vel purus mi. Curabitur vulputate dui id velit maximus, a aliquam lacus sodales. Sed condimentum, diam sed fringilla blandit, urna ante cursus quam, quis consectetur quam ipsum in lectus. Donec libero magna, hendrerit sit amet eros id, tempor venenatis nisi. Mauris ac lacus faucibus, dignissim neque id, tincidunt nisi. Pellentesque iaculis leo ut elit pulvinar tristique. Nam enim sapien, posuere vitae gravida gravida, volutpat ac ante. Vestibulum mattis mollis lorem, ut accumsan quam bibendum sed. Pellentesque sodales suscipit lorem, laoreet vehicula velit imperdiet eu. Aenean malesuada tortor eu ligula condimentum consequat. Aliquam at erat diam. Ut in nisi condimentum, faucibus odio ac, dapibus odio. Duis tempor dolor eu mauris euismod, at rhoncus leo luctus. \\


\section{Desenvolvimento M�vel Android}

Lorem ipsum dolor sit amet, consectetur adipiscing elit. Cras sodales turpis dolor, in porta justo sollicitudin eget. Duis sodales scelerisque viverra. Donec vulputate quam non diam ultricies, nec mattis mauris aliquam. Donec dictum dui ac dictum bibendum. Nunc placerat lobortis euismod. In pellentesque lectus imperdiet eros elementum, et viverra sapien fermentum. Aenean ultricies eu nulla eu molestie. Vivamus in vestibulum eros, rhoncus laoreet magna. Etiam imperdiet vestibulum metus, ac eleifend augue ornare cursus. Pellentesque vitae est purus. Ut eget ex libero. Mauris et sem iaculis, aliquet quam eget, feugiat nisl. Etiam porttitor sollicitudin facilisis. Donec egestas, diam sed tristique vestibulum, leo dolor cursus dui, non placerat lacus erat a augue. Nam ac consequat lectus.

Pellentesque rhoncus lacinia varius. Aliquam consectetur bibendum risus, in egestas tellus. Integer faucibus erat dui, ac interdum risus sodales in. Aenean vel purus mi. Curabitur vulputate dui id velit maximus, a aliquam lacus sodales. Sed condimentum, diam sed fringilla blandit, urna ante cursus quam, quis consectetur quam ipsum in lectus. Donec libero magna, hendrerit sit amet eros id, tempor venenatis nisi. Mauris ac lacus faucibus, dignissim neque id, tincidunt nisi. Pellentesque iaculis leo ut elit pulvinar tristique. Nam enim sapien, posuere vitae gravida gravida, volutpat ac ante. Vestibulum mattis mollis lorem, ut accumsan quam bibendum sed. Pellentesque sodales suscipit lorem, laoreet vehicula velit imperdiet eu. Aenean malesuada tortor eu ligula condimentum consequat. Aliquam at erat diam. Ut in nisi condimentum, faucibus odio ac, dapibus odio. Duis tempor dolor eu mauris euismod, at rhoncus leo luctus. \\


\section{Code Smells}\index{codesmells}

Lorem ipsum dolor sit amet, consectetur adipiscing elit. Cras sodales turpis dolor, in porta justo sollicitudin eget. Duis sodales scelerisque viverra. Donec vulputate quam non diam ultricies, nec mattis mauris aliquam. Donec dictum dui ac dictum bibendum. Nunc placerat lobortis euismod. In pellentesque lectus imperdiet eros elementum, et viverra sapien fermentum. Aenean ultricies eu nulla eu molestie. Vivamus in vestibulum eros, rhoncus laoreet magna. Etiam imperdiet vestibulum metus, ac eleifend augue ornare cursus. Pellentesque vitae est purus. Ut eget ex libero. Mauris et sem iaculis, aliquet quam eget, feugiat nisl. Etiam porttitor sollicitudin facilisis. Donec egestas, diam sed tristique vestibulum, leo dolor cursus dui, non placerat lacus erat a augue. Nam ac consequat lectus.

Pellentesque rhoncus lacinia varius. Aliquam consectetur bibendum risus, in egestas tellus. Integer faucibus erat dui, ac interdum risus sodales in. Aenean vel purus mi. Curabitur vulputate dui id velit maximus, a aliquam lacus sodales. Sed condimentum, diam sed fringilla blandit, urna ante cursus quam, quis consectetur quam ipsum in lectus. Donec libero magna, hendrerit sit amet eros id, tempor venenatis nisi. Mauris ac lacus faucibus, dignissim neque id, tincidunt nisi. Pellentesque iaculis leo ut elit pulvinar tristique. Nam enim sapien, posuere vitae gravida gravida, volutpat ac ante. Vestibulum mattis mollis lorem, ut accumsan quam bibendum sed. Pellentesque sodales suscipit lorem, laoreet vehicula velit imperdiet eu. Aenean malesuada tortor eu ligula condimentum consequat. Aliquam at erat diam. Ut in nisi condimentum, faucibus odio ac, dapibus odio. Duis tempor dolor eu mauris euismod, at rhoncus leo luctus. \\ 
%%%%%%%%%%%%%%%%%%%%%%%%%%%%%%%%%%%%%%%%%%%%%%%%%%%%%%%%%%%%%%%%%%%%%%%
\setlength{\parindent}{0pt}
\setlength{\textheight}{22cm}
\setlength{\parskip}{0.2cm}

% Para aumentar o espa�amento entre as linhas
\linespread{1.2}
%%%%%%%%%%%%%%%%%%%%%%%%%%%%%%%%%%%%%%%%%%%%%%%%%%%%%%%%%%%%%%%%%%%%%%%

\chapter{Pesquisa}

Lorem ipsum dolor sit amet, consectetur adipiscing elit. Cras sodales turpis dolor, in porta justo sollicitudin eget. Duis sodales scelerisque viverra. Donec vulputate quam non diam ultricies, nec mattis mauris aliquam. Donec dictum dui ac dictum bibendum. Nunc placerat lobortis euismod. In pellentesque lectus imperdiet eros elementum, et viverra sapien fermentum. Aenean ultricies eu nulla eu molestie. Vivamus in vestibulum eros, rhoncus laoreet magna. Etiam imperdiet vestibulum metus, ac eleifend augue ornare cursus. Pellentesque vitae est purus. Ut eget ex libero. Mauris et sem iaculis, aliquet quam eget, feugiat nisl. Etiam porttitor sollicitudin facilisis. Donec egestas, diam sed tristique vestibulum, leo dolor cursus dui, non placerat lacus erat a augue. Nam ac consequat lectus.

Pellentesque rhoncus lacinia varius. Aliquam consectetur bibendum risus, in egestas tellus. Integer faucibus erat dui, ac interdum risus sodales in. Aenean vel purus mi. Curabitur vulputate dui id velit maximus, a aliquam lacus sodales. Sed condimentum, diam sed fringilla blandit, urna ante cursus quam, quis consectetur quam ipsum in lectus. Donec libero magna, hendrerit sit amet eros id, tempor venenatis nisi. Mauris ac lacus faucibus, dignissim neque id, tincidunt nisi. Pellentesque iaculis leo ut elit pulvinar tristique. Nam enim sapien, posuere vitae gravida gravida, volutpat ac ante. Vestibulum mattis mollis lorem, ut accumsan quam bibendum sed. Pellentesque sodales suscipit lorem, laoreet vehicula velit imperdiet eu. Aenean malesuada tortor eu ligula condimentum consequat. Aliquam at erat diam. Ut in nisi condimentum, faucibus odio ac, dapibus odio. Duis tempor dolor eu mauris euismod, at rhoncus leo luctus. \\

\subsection*{Subsection 1}\index{Lorem!ipsum}

Lorem ipsum dolor sit amet, consectetur adipiscing elit. Cras sodales turpis dolor, in porta justo sollicitudin eget. Duis sodales scelerisque viverra. Donec vulputate quam non diam ultricies, nec mattis mauris aliquam. Donec dictum dui ac dictum bibendum. Nunc placerat lobortis euismod. In pellentesque lectus imperdiet eros elementum, et viverra sapien fermentum. Aenean ultricies eu nulla eu molestie. Vivamus in vestibulum eros, rhoncus laoreet magna. Etiam imperdiet vestibulum metus, ac eleifend augue ornare cursus. Pellentesque vitae est purus. Ut eget ex libero. Mauris et sem iaculis, aliquet quam eget, feugiat nisl. Etiam porttitor sollicitudin facilisis. Donec egestas, diam sed tristique vestibulum, leo dolor cursus dui, non placerat lacus erat a augue. Nam ac consequat lectus.

Pellentesque rhoncus lacinia varius. Aliquam consectetur bibendum risus, in egestas tellus. Integer faucibus erat dui, ac interdum risus sodales in. Aenean vel purus mi. Curabitur vulputate dui id velit maximus, a aliquam lacus sodales. Sed condimentum, diam sed fringilla blandit, urna ante cursus quam, quis consectetur quam ipsum in lectus. Donec libero magna, hendrerit sit amet eros id, tempor venenatis nisi. Mauris ac lacus faucibus, dignissim neque id, tincidunt nisi. Pellentesque iaculis leo ut elit pulvinar tristique. Nam enim sapien, posuere vitae gravida gravida, volutpat ac ante. Vestibulum mattis mollis lorem, ut accumsan quam bibendum sed. Pellentesque sodales suscipit lorem, laoreet vehicula velit imperdiet eu. Aenean malesuada tortor eu ligula condimentum consequat. Aliquam at erat diam. Ut in nisi condimentum, faucibus odio ac, dapibus odio. Duis tempor dolor eu mauris euismod, at rhoncus leo luctus. \\
 

\subsection*{Subsection 2}\index{Lorem!perspiciatis}

Lorem ipsum dolor sit amet, consectetur adipiscing elit. Cras sodales turpis dolor, in porta justo sollicitudin eget. Duis sodales scelerisque viverra. Donec vulputate quam non diam ultricies, nec mattis mauris aliquam. Donec dictum dui ac dictum bibendum. Nunc placerat lobortis euismod. In pellentesque lectus imperdiet eros elementum, et viverra sapien fermentum. Aenean ultricies eu nulla eu molestie. Vivamus in vestibulum eros, rhoncus laoreet magna. Etiam imperdiet vestibulum metus, ac eleifend augue ornare cursus. Pellentesque vitae est purus. Ut eget ex libero. Mauris et sem iaculis, aliquet quam eget, feugiat nisl. Etiam porttitor sollicitudin facilisis. Donec egestas, diam sed tristique vestibulum, leo dolor cursus dui, non placerat lacus erat a augue. Nam ac consequat lectus.

Pellentesque rhoncus lacinia varius. Aliquam consectetur bibendum risus, in egestas tellus. Integer faucibus erat dui, ac interdum risus sodales in. Aenean vel purus mi. Curabitur vulputate dui id velit maximus, a aliquam lacus sodales. Sed condimentum, diam sed fringilla blandit, urna ante cursus quam, quis consectetur quam ipsum in lectus. Donec libero magna, hendrerit sit amet eros id, tempor venenatis nisi. Mauris ac lacus faucibus, dignissim neque id, tincidunt nisi. Pellentesque iaculis leo ut elit pulvinar tristique. Nam enim sapien, posuere vitae gravida gravida, volutpat ac ante. Vestibulum mattis mollis lorem, ut accumsan quam bibendum sed. Pellentesque sodales suscipit lorem, laoreet vehicula velit imperdiet eu. Aenean malesuada tortor eu ligula condimentum consequat. Aliquam at erat diam. Ut in nisi condimentum, faucibus odio ac, dapibus odio. Duis tempor dolor eu mauris euismod, at rhoncus leo luctus. \\


\section{Hip�teses}\index{hipoteses}

Lorem ipsum dolor sit amet, consectetur adipiscing elit. Cras sodales turpis dolor, in porta justo sollicitudin eget. Duis sodales scelerisque viverra. Donec vulputate quam non diam ultricies, nec mattis mauris aliquam. Donec dictum dui ac dictum bibendum. Nunc placerat lobortis euismod. In pellentesque lectus imperdiet eros elementum, et viverra sapien fermentum. Aenean ultricies eu nulla eu molestie. Vivamus in vestibulum eros, rhoncus laoreet magna. Etiam imperdiet vestibulum metus, ac eleifend augue ornare cursus. Pellentesque vitae est purus. Ut eget ex libero. Mauris et sem iaculis, aliquet quam eget, feugiat nisl. Etiam porttitor sollicitudin facilisis. Donec egestas, diam sed tristique vestibulum, leo dolor cursus dui, non placerat lacus erat a augue. Nam ac consequat lectus.

Pellentesque rhoncus lacinia varius. Aliquam consectetur bibendum risus, in egestas tellus. Integer faucibus erat dui, ac interdum risus sodales in. Aenean vel purus mi. Curabitur vulputate dui id velit maximus, a aliquam lacus sodales. Sed condimentum, diam sed fringilla blandit, urna ante cursus quam, quis consectetur quam ipsum in lectus. Donec libero magna, hendrerit sit amet eros id, tempor venenatis nisi. Mauris ac lacus faucibus, dignissim neque id, tincidunt nisi. Pellentesque iaculis leo ut elit pulvinar tristique. Nam enim sapien, posuere vitae gravida gravida, volutpat ac ante. Vestibulum mattis mollis lorem, ut accumsan quam bibendum sed. Pellentesque sodales suscipit lorem, laoreet vehicula velit imperdiet eu. Aenean malesuada tortor eu ligula condimentum consequat. Aliquam at erat diam. Ut in nisi condimentum, faucibus odio ac, dapibus odio. Duis tempor dolor eu mauris euismod, at rhoncus leo luctus. \\


\section{Processo de Pesquisa}

Lorem ipsum dolor sit amet, consectetur adipiscing elit. Cras sodales turpis dolor, in porta justo sollicitudin eget. Duis sodales scelerisque viverra. Donec vulputate quam non diam ultricies, nec mattis mauris aliquam. Donec dictum dui ac dictum bibendum. Nunc placerat lobortis euismod. In pellentesque lectus imperdiet eros elementum, et viverra sapien fermentum. Aenean ultricies eu nulla eu molestie. Vivamus in vestibulum eros, rhoncus laoreet magna. Etiam imperdiet vestibulum metus, ac eleifend augue ornare cursus. Pellentesque vitae est purus. Ut eget ex libero. Mauris et sem iaculis, aliquet quam eget, feugiat nisl. Etiam porttitor sollicitudin facilisis. Donec egestas, diam sed tristique vestibulum, leo dolor cursus dui, non placerat lacus erat a augue. Nam ac consequat lectus.

Pellentesque rhoncus lacinia varius. Aliquam consectetur bibendum risus, in egestas tellus. Integer faucibus erat dui, ac interdum risus sodales in. Aenean vel purus mi. Curabitur vulputate dui id velit maximus, a aliquam lacus sodales. Sed condimentum, diam sed fringilla blandit, urna ante cursus quam, quis consectetur quam ipsum in lectus. Donec libero magna, hendrerit sit amet eros id, tempor venenatis nisi. Mauris ac lacus faucibus, dignissim neque id, tincidunt nisi. Pellentesque iaculis leo ut elit pulvinar tristique. Nam enim sapien, posuere vitae gravida gravida, volutpat ac ante. Vestibulum mattis mollis lorem, ut accumsan quam bibendum sed. Pellentesque sodales suscipit lorem, laoreet vehicula velit imperdiet eu. Aenean malesuada tortor eu ligula condimentum consequat. Aliquam at erat diam. Ut in nisi condimentum, faucibus odio ac, dapibus odio. Duis tempor dolor eu mauris euismod, at rhoncus leo luctus. \\


\subsection{Coleta de Dados}

Lorem ipsum dolor sit amet, consectetur adipiscing elit. Cras sodales turpis dolor, in porta justo sollicitudin eget. Duis sodales scelerisque viverra. Donec vulputate quam non diam ultricies, nec mattis mauris aliquam. Donec dictum dui ac dictum bibendum. Nunc placerat lobortis euismod. In pellentesque lectus imperdiet eros elementum, et viverra sapien fermentum. Aenean ultricies eu nulla eu molestie. Vivamus in vestibulum eros, rhoncus laoreet magna. Etiam imperdiet vestibulum metus, ac eleifend augue ornare cursus. Pellentesque vitae est purus. Ut eget ex libero. Mauris et sem iaculis, aliquet quam eget, feugiat nisl. Etiam porttitor sollicitudin facilisis. Donec egestas, diam sed tristique vestibulum, leo dolor cursus dui, non placerat lacus erat a augue. Nam ac consequat lectus.

Pellentesque rhoncus lacinia varius. Aliquam consectetur bibendum risus, in egestas tellus. Integer faucibus erat dui, ac interdum risus sodales in. Aenean vel purus mi. Curabitur vulputate dui id velit maximus, a aliquam lacus sodales. Sed condimentum, diam sed fringilla blandit, urna ante cursus quam, quis consectetur quam ipsum in lectus. Donec libero magna, hendrerit sit amet eros id, tempor venenatis nisi. Mauris ac lacus faucibus, dignissim neque id, tincidunt nisi. Pellentesque iaculis leo ut elit pulvinar tristique. Nam enim sapien, posuere vitae gravida gravida, volutpat ac ante. Vestibulum mattis mollis lorem, ut accumsan quam bibendum sed. Pellentesque sodales suscipit lorem, laoreet vehicula velit imperdiet eu. Aenean malesuada tortor eu ligula condimentum consequat. Aliquam at erat diam. Ut in nisi condimentum, faucibus odio ac, dapibus odio. Duis tempor dolor eu mauris euismod, at rhoncus leo luctus. \\

\subsection{An�lise dos Dados}

Lorem ipsum dolor sit amet, consectetur adipiscing elit. Cras sodales turpis dolor, in porta justo sollicitudin eget. Duis sodales scelerisque viverra. Donec vulputate quam non diam ultricies, nec mattis mauris aliquam. Donec dictum dui ac dictum bibendum. Nunc placerat lobortis euismod. In pellentesque lectus imperdiet eros elementum, et viverra sapien fermentum. Aenean ultricies eu nulla eu molestie. Vivamus in vestibulum eros, rhoncus laoreet magna. Etiam imperdiet vestibulum metus, ac eleifend augue ornare cursus. Pellentesque vitae est purus. Ut eget ex libero. Mauris et sem iaculis, aliquet quam eget, feugiat nisl. Etiam porttitor sollicitudin facilisis. Donec egestas, diam sed tristique vestibulum, leo dolor cursus dui, non placerat lacus erat a augue. Nam ac consequat lectus.

Pellentesque rhoncus lacinia varius. Aliquam consectetur bibendum risus, in egestas tellus. Integer faucibus erat dui, ac interdum risus sodales in. Aenean vel purus mi. Curabitur vulputate dui id velit maximus, a aliquam lacus sodales. Sed condimentum, diam sed fringilla blandit, urna ante cursus quam, quis consectetur quam ipsum in lectus. Donec libero magna, hendrerit sit amet eros id, tempor venenatis nisi. Mauris ac lacus faucibus, dignissim neque id, tincidunt nisi. Pellentesque iaculis leo ut elit pulvinar tristique. Nam enim sapien, posuere vitae gravida gravida, volutpat ac ante. Vestibulum mattis mollis lorem, ut accumsan quam bibendum sed. Pellentesque sodales suscipit lorem, laoreet vehicula velit imperdiet eu. Aenean malesuada tortor eu ligula condimentum consequat. Aliquam at erat diam. Ut in nisi condimentum, faucibus odio ac, dapibus odio. Duis tempor dolor eu mauris euismod, at rhoncus leo luctus. \\


\section{Escrita dos Code Smells}

Lorem ipsum dolor sit amet, consectetur adipiscing elit. Cras sodales turpis dolor, in porta justo sollicitudin eget. Duis sodales scelerisque viverra. Donec vulputate quam non diam ultricies, nec mattis mauris aliquam. Donec dictum dui ac dictum bibendum. Nunc placerat lobortis euismod. In pellentesque lectus imperdiet eros elementum, et viverra sapien fermentum. Aenean ultricies eu nulla eu molestie. Vivamus in vestibulum eros, rhoncus laoreet magna. Etiam imperdiet vestibulum metus, ac eleifend augue ornare cursus. Pellentesque vitae est purus. Ut eget ex libero. Mauris et sem iaculis, aliquet quam eget, feugiat nisl. Etiam porttitor sollicitudin facilisis. Donec egestas, diam sed tristique vestibulum, leo dolor cursus dui, non placerat lacus erat a augue. Nam ac consequat lectus.

Pellentesque rhoncus lacinia varius. Aliquam consectetur bibendum risus, in egestas tellus. Integer faucibus erat dui, ac interdum risus sodales in. Aenean vel purus mi. Curabitur vulputate dui id velit maximus, a aliquam lacus sodales. Sed condimentum, diam sed fringilla blandit, urna ante cursus quam, quis consectetur quam ipsum in lectus. Donec libero magna, hendrerit sit amet eros id, tempor venenatis nisi. Mauris ac lacus faucibus, dignissim neque id, tincidunt nisi. Pellentesque iaculis leo ut elit pulvinar tristique. Nam enim sapien, posuere vitae gravida gravida, volutpat ac ante. Vestibulum mattis mollis lorem, ut accumsan quam bibendum sed. Pellentesque sodales suscipit lorem, laoreet vehicula velit imperdiet eu. Aenean malesuada tortor eu ligula condimentum consequat. Aliquam at erat diam. Ut in nisi condimentum, faucibus odio ac, dapibus odio. Duis tempor dolor eu mauris euismod, at rhoncus leo luctus. \\


 
%%%%%%%%%%%%%%%%%%%%%%%%%%%%%%%%%%%%%%%%%%%%%%%%%%%%%%%%%%%%%%%%%%%%%%%
\setlength{\parindent}{0pt}
\setlength{\textheight}{22cm}
\setlength{\parskip}{0.2cm}

% Para aumentar o espa�amento entre as linhas
\linespread{1.2}
%%%%%%%%%%%%%%%%%%%%%%%%%%%%%%%%%%%%%%%%%%%%%%%%%%%%%%%%%%%%%%%%%%%%%%%

\chapter{Cat�logo de Code Smells}

Planejamento\index{planejamento} � um componente importante do comportamento racional pois trata-se de um processo de s�ntese que seleciona e organiza a��es antecipando seus efeitos. Este processo busca satisfazer, da melhor forma poss�vel, um conjunto de metas pr�-definidas. \\

Muitas tarefas\index{tarefa} humanas necessitam de planejamento. Lorem ipsum dolor sit amet, consectetur adipiscing elit, sed do eiusmod tempor incididunt ut labore et dolore magna aliqua. Ut enim ad minim veniam, quis nostrud exercitation ullamco laboris nisi ut aliquip ex ea commodo consequat. Duis aute irure dolor in reprehenderit in voluptate velit esse cillum dolore eu fugiat nulla pariatur. Excepteur sint occaecat cupidatat non proident, sunt in culpa qui officia deserunt mollit anim id est laborum. \cite{RomanWeerdt2004}. Lorem ipsum dolor sit amet, consectetur adipiscing elit, sed do eiusmod tempor incididunt ut labore et dolore magna aliqua. O Exemplo \ref{ex:viagem_curitiba} ilustra um problema real de planejamento. \\


\section{Code Smell 1}

Lorem ipsum dolor sit amet, consectetur adipiscing elit. Cras sodales turpis dolor, in porta justo sollicitudin eget. Duis sodales scelerisque viverra. Donec vulputate quam non diam ultricies, nec mattis mauris aliquam. Donec dictum dui ac dictum bibendum. Nunc placerat lobortis euismod. In pellentesque lectus imperdiet eros elementum, et viverra sapien fermentum. Aenean ultricies eu nulla eu molestie. Vivamus in vestibulum eros, rhoncus laoreet magna. Etiam imperdiet vestibulum metus, ac eleifend augue ornare cursus. Pellentesque vitae est purus. Ut eget ex libero. Mauris et sem iaculis, aliquet quam eget, feugiat nisl. Etiam porttitor sollicitudin facilisis. Donec egestas, diam sed tristique vestibulum, leo dolor cursus dui, non placerat lacus erat a augue. Nam ac consequat lectus.

Pellentesque rhoncus lacinia varius. Aliquam consectetur bibendum risus, in egestas tellus. Integer faucibus erat dui, ac interdum risus sodales in. Aenean vel purus mi. Curabitur vulputate dui id velit maximus, a aliquam lacus sodales. Sed condimentum, diam sed fringilla blandit, urna ante cursus quam, quis consectetur quam ipsum in lectus. Donec libero magna, hendrerit sit amet eros id, tempor venenatis nisi. Mauris ac lacus faucibus, dignissim neque id, tincidunt nisi. Pellentesque iaculis leo ut elit pulvinar tristique. Nam enim sapien, posuere vitae gravida gravida, volutpat ac ante. Vestibulum mattis mollis lorem, ut accumsan quam bibendum sed. Pellentesque sodales suscipit lorem, laoreet vehicula velit imperdiet eu. Aenean malesuada tortor eu ligula condimentum consequat. Aliquam at erat diam. Ut in nisi condimentum, faucibus odio ac, dapibus odio. Duis tempor dolor eu mauris euismod, at rhoncus leo luctus. \\


\section{Code Smell 2}

Lorem ipsum dolor sit amet, consectetur adipiscing elit. Cras sodales turpis dolor, in porta justo sollicitudin eget. Duis sodales scelerisque viverra. Donec vulputate quam non diam ultricies, nec mattis mauris aliquam. Donec dictum dui ac dictum bibendum. Nunc placerat lobortis euismod. In pellentesque lectus imperdiet eros elementum, et viverra sapien fermentum. Aenean ultricies eu nulla eu molestie. Vivamus in vestibulum eros, rhoncus laoreet magna. Etiam imperdiet vestibulum metus, ac eleifend augue ornare cursus. Pellentesque vitae est purus. Ut eget ex libero. Mauris et sem iaculis, aliquet quam eget, feugiat nisl. Etiam porttitor sollicitudin facilisis. Donec egestas, diam sed tristique vestibulum, leo dolor cursus dui, non placerat lacus erat a augue. Nam ac consequat lectus.

Pellentesque rhoncus lacinia varius. Aliquam consectetur bibendum risus, in egestas tellus. Integer faucibus erat dui, ac interdum risus sodales in. Aenean vel purus mi. Curabitur vulputate dui id velit maximus, a aliquam lacus sodales. Sed condimentum, diam sed fringilla blandit, urna ante cursus quam, quis consectetur quam ipsum in lectus. Donec libero magna, hendrerit sit amet eros id, tempor venenatis nisi. Mauris ac lacus faucibus, dignissim neque id, tincidunt nisi. Pellentesque iaculis leo ut elit pulvinar tristique. Nam enim sapien, posuere vitae gravida gravida, volutpat ac ante. Vestibulum mattis mollis lorem, ut accumsan quam bibendum sed. Pellentesque sodales suscipit lorem, laoreet vehicula velit imperdiet eu. Aenean malesuada tortor eu ligula condimentum consequat. Aliquam at erat diam. Ut in nisi condimentum, faucibus odio ac, dapibus odio. Duis tempor dolor eu mauris euismod, at rhoncus leo luctus. \\


\section{Code Smell 3}

Lorem ipsum dolor sit amet, consectetur adipiscing elit. Cras sodales turpis dolor, in porta justo sollicitudin eget. Duis sodales scelerisque viverra. Donec vulputate quam non diam ultricies, nec mattis mauris aliquam. Donec dictum dui ac dictum bibendum. Nunc placerat lobortis euismod. In pellentesque lectus imperdiet eros elementum, et viverra sapien fermentum. Aenean ultricies eu nulla eu molestie. Vivamus in vestibulum eros, rhoncus laoreet magna. Etiam imperdiet vestibulum metus, ac eleifend augue ornare cursus. Pellentesque vitae est purus. Ut eget ex libero. Mauris et sem iaculis, aliquet quam eget, feugiat nisl. Etiam porttitor sollicitudin facilisis. Donec egestas, diam sed tristique vestibulum, leo dolor cursus dui, non placerat lacus erat a augue. Nam ac consequat lectus.

Pellentesque rhoncus lacinia varius. Aliquam consectetur bibendum risus, in egestas tellus. Integer faucibus erat dui, ac interdum risus sodales in. Aenean vel purus mi. Curabitur vulputate dui id velit maximus, a aliquam lacus sodales. Sed condimentum, diam sed fringilla blandit, urna ante cursus quam, quis consectetur quam ipsum in lectus. Donec libero magna, hendrerit sit amet eros id, tempor venenatis nisi. Mauris ac lacus faucibus, dignissim neque id, tincidunt nisi. Pellentesque iaculis leo ut elit pulvinar tristique. Nam enim sapien, posuere vitae gravida gravida, volutpat ac ante. Vestibulum mattis mollis lorem, ut accumsan quam bibendum sed. Pellentesque sodales suscipit lorem, laoreet vehicula velit imperdiet eu. Aenean malesuada tortor eu ligula condimentum consequat. Aliquam at erat diam. Ut in nisi condimentum, faucibus odio ac, dapibus odio. Duis tempor dolor eu mauris euismod, at rhoncus leo luctus. \\


\section{Code Smell 4}

Lorem ipsum dolor sit amet, consectetur adipiscing elit. Cras sodales turpis dolor, in porta justo sollicitudin eget. Duis sodales scelerisque viverra. Donec vulputate quam non diam ultricies, nec mattis mauris aliquam. Donec dictum dui ac dictum bibendum. Nunc placerat lobortis euismod. In pellentesque lectus imperdiet eros elementum, et viverra sapien fermentum. Aenean ultricies eu nulla eu molestie. Vivamus in vestibulum eros, rhoncus laoreet magna. Etiam imperdiet vestibulum metus, ac eleifend augue ornare cursus. Pellentesque vitae est purus. Ut eget ex libero. Mauris et sem iaculis, aliquet quam eget, feugiat nisl. Etiam porttitor sollicitudin facilisis. Donec egestas, diam sed tristique vestibulum, leo dolor cursus dui, non placerat lacus erat a augue. Nam ac consequat lectus.

Pellentesque rhoncus lacinia varius. Aliquam consectetur bibendum risus, in egestas tellus. Integer faucibus erat dui, ac interdum risus sodales in. Aenean vel purus mi. Curabitur vulputate dui id velit maximus, a aliquam lacus sodales. Sed condimentum, diam sed fringilla blandit, urna ante cursus quam, quis consectetur quam ipsum in lectus. Donec libero magna, hendrerit sit amet eros id, tempor venenatis nisi. Mauris ac lacus faucibus, dignissim neque id, tincidunt nisi. Pellentesque iaculis leo ut elit pulvinar tristique. Nam enim sapien, posuere vitae gravida gravida, volutpat ac ante. Vestibulum mattis mollis lorem, ut accumsan quam bibendum sed. Pellentesque sodales suscipit lorem, laoreet vehicula velit imperdiet eu. Aenean malesuada tortor eu ligula condimentum consequat. Aliquam at erat diam. Ut in nisi condimentum, faucibus odio ac, dapibus odio. Duis tempor dolor eu mauris euismod, at rhoncus leo luctus. \\


\section{Code Smell 5}

Lorem ipsum dolor sit amet, consectetur adipiscing elit. Cras sodales turpis dolor, in porta justo sollicitudin eget. Duis sodales scelerisque viverra. Donec vulputate quam non diam ultricies, nec mattis mauris aliquam. Donec dictum dui ac dictum bibendum. Nunc placerat lobortis euismod. In pellentesque lectus imperdiet eros elementum, et viverra sapien fermentum. Aenean ultricies eu nulla eu molestie. Vivamus in vestibulum eros, rhoncus laoreet magna. Etiam imperdiet vestibulum metus, ac eleifend augue ornare cursus. Pellentesque vitae est purus. Ut eget ex libero. Mauris et sem iaculis, aliquet quam eget, feugiat nisl. Etiam porttitor sollicitudin facilisis. Donec egestas, diam sed tristique vestibulum, leo dolor cursus dui, non placerat lacus erat a augue. Nam ac consequat lectus.

Pellentesque rhoncus lacinia varius. Aliquam consectetur bibendum risus, in egestas tellus. Integer faucibus erat dui, ac interdum risus sodales in. Aenean vel purus mi. Curabitur vulputate dui id velit maximus, a aliquam lacus sodales. Sed condimentum, diam sed fringilla blandit, urna ante cursus quam, quis consectetur quam ipsum in lectus. Donec libero magna, hendrerit sit amet eros id, tempor venenatis nisi. Mauris ac lacus faucibus, dignissim neque id, tincidunt nisi. Pellentesque iaculis leo ut elit pulvinar tristique. Nam enim sapien, posuere vitae gravida gravida, volutpat ac ante. Vestibulum mattis mollis lorem, ut accumsan quam bibendum sed. Pellentesque sodales suscipit lorem, laoreet vehicula velit imperdiet eu. Aenean malesuada tortor eu ligula condimentum consequat. Aliquam at erat diam. Ut in nisi condimentum, faucibus odio ac, dapibus odio. Duis tempor dolor eu mauris euismod, at rhoncus leo luctus. \\ 
% -*- root: dissertacao.tex -*-
%%%%%%%%%%%%%%%%%%%%%%%%%%%%%%%%%%%%%%%%%%%%%%%%%%%%%%%%%%%%%%%%%%%%%%%
\setlength{\parindent}{0pt}
\setlength{\textheight}{22cm}
\setlength{\parskip}{0.2cm}

% Para aumentar o espa�amento entre as linhas
\linespread{1.2}
%%%%%%%%%%%%%%%%%%%%%%%%%%%%%%%%%%%%%%%%%%%%%%%%%%%%%%%%%%%%%%%%%%%%%%%

\chapter{Conclus�o}
\label{cap:conclusao}

A fazer. \\


\section{Principais contribui��es}

A fazer. \\


\section{Trabalhos futuros}

A fazer.

%\appendix

%%%%%%%%%%%%%%%%%%%%%%%%%%%%%%%%%%%%%%%%%%%%%%%%%%%%%%%%%%%%%%%%%%%%%%%%
\setlength{\parindent}{0pt}
\setlength{\textheight}{22cm}
\setlength{\parskip}{0.2cm}

% Para aumentar o espa�amento entre as linhas
\linespread{1.2}
%%%%%%%%%%%%%%%%%%%%%%%%%%%%%%%%%%%%%%%%%%%%%%%%%%%%%%%%%%%%%%%%%%%%%%%

\chapter{XYZ}

\section{Ap�ndice 1}

A fazer. \\


%%%%%%%%%%%%%%%%%%%%%%%%%%%%%%%%%%%%%%%%%%%%%%%%%%%%%%%%%%%%%%%%%%%%%%%%%

\onehalfspacing
\bibliographystyle{alpha}
\bibliography{bibliografias}
\bibliography{library}

\printindex

\end{document}
