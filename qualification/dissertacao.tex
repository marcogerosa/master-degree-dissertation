\documentclass[12pt,letterpaper,oneside]{book} 
%\documentclass[12pt,twoside,letterpaper]{book}
% oneside indica que nao � frente e verso

% ---------------------------------------------------------------------------- %
% Pacotes 
\usepackage[T1]{fontenc}
\usepackage[brazil]{babel}
%\usepackage[latin1]{inputenc}
\usepackage[pdftex]{graphicx}           % usamos arquivos pdf/png como figuras
\usepackage{setspace}                   % espa�amento flex�vel
\usepackage{indentfirst}                % indenta��o do primeiro par�grafo
\usepackage{makeidx}                    % �ndice remissivo
\usepackage[nottoc]{tocbibind}          % acrescentamos a bibliografia/indice/conteudo no Table of Contents
\usepackage{courier}                    % usa o Adobe Courier no lugar de Computer Modern Typewriter
\usepackage{type1cm}                    % fontes realmente escal�veis
\usepackage{listings}                   % para formatar c�digo-fonte (ex. em Java)

\usepackage{titletoc}
\usepackage{booktabs}                   % para gera��o de tabelas
\usepackage[bf,small,compact]{titlesec} % cabe�alhos dos t�tulos: menores e compactos
\usepackage[fixlanguage]{babelbib}
\usepackage[font=small,format=plain,labelfont=bf,up,textfont=it,up]{caption}
\usepackage[usenames,svgnames,dvipsnames]{xcolor}
\usepackage[a4paper,top=2.54cm,bottom=2.0cm,left=2.0cm,right=2.54cm]{geometry} % margens
\usepackage[pdftex,plainpages=false,pdfpagelabels,pagebackref,colorlinks=true,citecolor=black,linkcolor=black,urlcolor=black,filecolor=black,bookmarksopen=true]{hyperref} % links em preto
% \usepackage[pdftex,plainpages=false,pdfpagelabels,pagebackref,colorlinks=true,citecolor=DarkGreen,linkcolor=NavyBlue,urlcolor=DarkRed,filecolor=green,bookmarksopen=true]{hyperref} % links coloridos
\usepackage[all]{hypcap}                    % soluciona o problema com o hyperref e capitulos
%\usepackage[square,sort,nonamebreak,comma]{natbib}  % cita��o bibliogr�fica alpha (alpha-ime.bst)

% By David
\usepackage{amsthm}
\usepackage{acronym} 
% \usepackage[portugues,ruled,vlined,linesnumbered]{algorithm2e/algorithm2e}
\usepackage{supertabular}

% By Suelen
\usepackage{gensymb}
\usepackage[bottom]{footmisc}

% ---------------------------------------------------------------------------- %
% Cabe�alhos similares ao TAOCP de Donald E. Knuth
\usepackage{fancyhdr}
\pagestyle{fancy}
\fancyhf{}
\renewcommand{\lstlistingname}{C�digo-fonte}
\renewcommand{\chaptermark}[1]{\markboth{\MakeUppercase{#1}}{}}
\renewcommand{\sectionmark}[1]{\markright{\MakeUppercase{#1}}{}}
\renewcommand{\headrulewidth}{0pt}

% ---------------------------------------------------------------------------- %
\graphicspath{{./figuras/}}             % caminho das figuras (recomend�vel)
\frenchspacing                          % arruma o espa�o: id est (i.e.) e exempli gratia (e.g.) 
\urlstyle{same}                         % URL com o mesmo estilo do texto e n�o mono-spaced
\makeindex                              % para o �ndice remissivo
\raggedbottom                           % para n�o permitir espa�os extra no texto
\fontsize{60}{62}\usefont{OT1}{cmr}{m}{n}{\selectfont}
\cleardoublepage
\normalsize

% ---------------------------------------------------------------------------- %
% Op��es de listing usados para o c�digo fonte
% Ref: http://en.wikibooks.org/wiki/LaTeX/Packages/Listings
\lstset{ %
language=Java,                  % choose the language of the code
basicstyle=\footnotesize,       % the size of the fonts that are used for the code
numbers=left,                   % where to put the line-numbers
numberstyle=\footnotesize,      % the size of the fonts that are used for the line-numbers
stepnumber=1,                   % the step between two line-numbers. If it's 1 each line will be numbered
numbersep=5pt,                  % how far the line-numbers are from the code
showspaces=false,               % show spaces adding particular underscores
showstringspaces=false,         % underline spaces within strings
showtabs=false,                 % show tabs within strings adding particular underscores
frame=single,	                % adds a frame around the code
framerule=0.6pt,
tabsize=2,	                    % sets default tabsize to 2 spaces
captionpos=b,                   % sets the caption-position to bottom
breaklines=true,                % sets automatic line breaking
breakatwhitespace=false,        % sets if automatic breaks should only happen at whitespace
escapeinside={\%*}{*)},         % if you want to add a comment within your code
backgroundcolor=\color[rgb]{1.0,1.0,1.0}, % choose the background color.
rulecolor=\color[rgb]{0.8,0.8,0.8},
extendedchars=true,
xleftmargin=\parindent,
xrightmargin=\parindent,
framexleftmargin=15pt,
framexrightmargin=15pt
}


\pagestyle{headings}
\markboth{}{}

% ---------------------------------------------------------------------------- %
% Dimens�es da p�gina (letterpaper)
%\setlength{\paperwidth}{216mm}
%\setlength{\topmargin}{1.3cm}         % deslocamento do topo do texto 
%\setlength\oddsidemargin{0cm}
%\setlength\evensidemargin{0cm}
%\setlength{\parskip}{1.2mm}
%\setlength{\parindent}{4mm}
%\setlength{\textwidth}{135mm}          % largura do texto
%\setlength{\parindent}{0pt}
%\setlength{\textheight}{22cm}
%\setlength{\parskip}{0.2cm}


\newcommand{\eb}{\varepsilon}
\newcommand{\mdp}{\langle\mathcal{S,A},p,r,c\rangle}
\newcommand{\ctlstar}{{\sc ctl}$^\star$}
\newcommand{\ctl}{\sc ctl}
\newcommand{\ltl}{\sc ltl}
\newtheorem{Def}{Defini��o}[chapter]
\newtheorem{Teo}{Teorema}[chapter]
\newtheorem{Ex}{Exemplo}[section]
\newtheorem{Tab}{Tabela}[chapter]

\begin{document}
%\hypersetup{
%pdfauthor = {Suelen Goularte Carvalho},
%pdftitle = {Algo Relacionado a Mobile},
%pdfsubject = {Disserta��o de Mestrado},
%pdfkeywords={Mobile, Computa��o Movel, Celular} % <== Precisa rever o que vai colocar aqui !!!
%pdfcreator = {LaTeX with hyperref package},
%}

\frontmatter

\onehalfspacing
\thispagestyle{empty}
\begin{center}
    \vspace*{0.2cm}
    \textbf{\Large{Um conjunto validade de maus cheiros na aplica��o do padr�o Model-View-Presenter no dom�nio de Aplica��es Android Nativas a ser apresentado � CPG para a disserta��o}}\\
	
    \vspace*{1.2cm}
    \Large{Suelen Goularte Carvalho} \\ 
    
    \vskip 2cm
	\textsc{
	Disserta��o apresentada\\[-0.25cm] 
	ao\\[-0.25cm]
	Instituto de Matem�tica e Estat�stica\\[-0.25cm]
	da\\[-0.25cm]
	Universidade de S�o Paulo\\[-0.25cm]
	para\\[-0.25cm]
	obten��o do t�tulo\\[-0.25cm]
	de\\[-0.25cm]
	Mestre em Ci�ncias}
    
    \vskip 1.5cm
    Programa: Ci�ncia da Computa��o\\
    Orientador: Prof. Dr. Marco Aur�lio Gerosa\\

    \vskip 1.5cm
    �rea de Concentra��o: Computa��o M�vel\\
    Orientador: Prof. Dr. Marco Aur�lio Gerosa\\

    \vskip 1cm
	\normalsize{}
	
    \vskip 0.5cm
    \normalsize{S�o Paulo, Julho de 2016}

\end{center}

% P�gina de rosto
\newpage
\thispagestyle{empty}
	\begin{center}
    	\vspace*{0.2 cm}
        \textbf{\Large{Um conjunto validade de maus cheiros na aplica��o do padr�o Model-View-Presenter no dom�nio de Aplica��es Android Nativas a ser apresentado � CPG para disserta��o}}\\
	    \vspace*{2 cm}
	\end{center}

	\vskip 2cm

	\begin{flushright}
	Esta � a vers�o original da disserta��o elaborada pelo\\
	candidato Suelen Goularte Carvalho, tal como\\
	submetida a Comiss�o Julgadora.\\
	\vskip 3cm

	\end{flushright}
	\vskip 4.2cm

	\begin{quote}
	\noindent Comiss�o Julgadora:
	
	\begin{itemize}
		\item {Prof. Dr. Marco Aur�lio Gerosa $-$ IME-USP}
		\item {Prof. Dr. Nome 222 222 $-$ IME-USP}
		\item {Prof. Dr. Nome 333 333 $-$ IME-USP}
	\end{itemize}
	  
	\end{quote}

\newpage
\thispagestyle{empty}
	\vspace*{12cm}
	\vskip 1cm

	\begin{flushright}
	{\small Dedico esta disserta��o de mestrado aos meus\\
	\ldots \\
	\ldots \\
	\ldots .\\}
	\end{flushright}

	\vspace*{1cm}

	\begin{flushright}
	%{\it ``A Estrada vai sempre em frente''} \\
	%$-$ Bilbo Baggins
	\end{flushright}

\pagebreak


\pagenumbering{roman}

\onehalfspacing
% \chapter*{Agradecimentos}
\setlength{\parindent}{0mm}

A fazer. \\

% -*- root: dissertation.tex -*-
\noindent CARVALHO, G. S. \textbf{Anomalies in the Presentation Layer of Android Applications}. 
2018. %100 f.
Disserta��o (Mestrado) - Instituto de Matem�tica e Estat�stica,
Universidade de S�o Paulo, S�o Paulo, 2018.
\\

We are aware that good codes matter, but how do you know when quality is low? Code smells help us identify problematic code snippets, but most of the code smells cataloged are based on traditional practices and technologies, created from the 1970s through the 90s, such as object oriented and Java. There are still doubts about code smells in technologies that have emerged in the last decade, such as Android, the main mobile platform in 2017 with more than 86\% market share. Some researchers have defined code smells related to Android efficiency and usability. Other research concludes that the components most affected by traditional code smells are related to the front-end, such as \textit{Activities} and \textit{Adapters}. Also noticed in some applications, front-end's codes represent the larger part of the project's code. It is noteworthy that the Android front-end is also composed of XML files, called application resources, used to build user interface (UI), but these files were not considered in their analyzes. In this dissertation, we investigate existence of code smells related to the Android front-end considering even application resources. To aim that performed two online surveys and three experiments summing 3XX developers. Our results show that there is a common perception among practicing Android developers about bad practices on Android front-end. Our main contributions are a catalog of 21 code smells about Android front-end and a statistical analysis of the perceptions of practitioners developers about the main 8 code smells cataloged. Our contributions will serve researchers as a starting point for the definition of heuristics and implementation of automated tools and to practitioners developers as an aid in identifying problematic codes, even manually.


\noindent \textbf{Palavras-chave:} software engineering, android, code smells, code quality, software maintenance, software anomalies.

\onehalfspacing
\tableofcontents

\chapter{Lista de Abreviaturas}

\begin{acronym}

\acro{SDK}{{\it Software Development Kit}} % 
\acro{IDE}{{\it Integrated Development Environment}} % 
\acro{APK}{{\it Android Package}} % 
\acro{ART}{{\it Android RunTime}} % 

\end{acronym}


% \include{nomenclaturas}

\listoffigures
% \listofalgorithms

\mainmatter

%%%%%%%%%%%%%%%%%%%%%%%%%%%%%%%%%%%%%%%%%%%%%%%%%%%%%%%%%%%%%%%%%%%%%%%%%
\onehalfspacing

% -*- root: dissertacao.tex -*-
%%%%%%%%%%%%%%%%%%%%%%%%%%%%%%%%%%%%%%%%%%%%%%%%%%%%%%%%%%%%%%%%%%%%%%%
\setlength{\parindent}{20pt}
\setlength{\textheight}{22cm}
\setlength{\parskip}{0.2cm}
\linespread{1.2} % Para aumentar o espa�amento entre as linhas
%%%%%%%%%%%%%%%%%%%%%%%%%%%%%%%%%%%%%%%%%%%%%%%%%%%%%%%%%%%%%%%%%%%%%%%

% Identificar o contexto e a motiva��o para o trabalho, especificar o problema, discutir trabalhos relacionados (principalmente as suas limita��es), definir as contribui��es, enfatizar os resultados principais e a organiza��o do texto.

\chapter{Introdu��o}

Em 2017 o Android completar� uma d�cada desde seu primeiro lan�amento em 2007. Atualmente h� dispon�vel mais de 2 milh�es de aplicativos na Google Play Store, loja oficial de aplicativos Android \cite{AppInPlayStore:09-16}. Mais de 83,5\% dos dispositivos m�veis no mundo usam o sistema operacional Android, e esse percentual vem crescendo ano ap�s ano \cite{MobileMarketShares:16, GrowthForecastIDC:16}. Atualmente � poss�vel encontr�-lo tamb�m em outros dispositivos como, \textit{smart TVs}, \textit{smartwatchs}, carros, dentre outros \cite{AndroidSmart:14, AndroidAuto:16}. 

Aplicativos Android t�m se tornado complexos projetos de software que precisam ser rapidamente desenvolvidos e regularmente evolu�dos para atender aos requisitos dos usu�rios. Esse contexto pode levar a decis�es ruins de \textit{design} de c�digo conhecidas como anomalias ou maus cheiros, e podem degradar a qualidade do projeto tornando-o de dif�cil manuten��o e evolu��o \cite{Hecht2015}. Apesar de ser poss�vel analisar projetos Android atrav�s de maus cheiros ``tradicionais'' (por exemplo, \textit{God Classes} e \textit{Long Methods}), pesquisas t�m demonstrado que diferentes plataformas, linguagens e frameworks podem apresentar m�tricas de qualidade de c�digo espec�ficas \cite{FinavaroAniche2016, Zhang2013}. Projetos Android possuem caracter�sticas espec�ficas principalmente com rela��o � camada de apresenta��o. 

% Uma tarefa comum a desenvolvedores de software � detectar trechos de c�digo problem�ticos. Para isso � comum se utilizar de estrat�gias de detec��o de maus cheiros \cite{FinavaroAniche2016}. Maus cheiros s�o s�ntomas de escolhas ruins de design e implementa��o de c�digo \cite{FinavaroAniche2016}. Apesar de j� existir um cat�logo extenso de maus cheiros, eles n�o levam em considera��o a natureza do projeto e suas particularidades. 

% Constantemente desenvolvedores de software i) escrevem c�digo f�cil de ser mantido e evolu�do e ii) detectam trechos de c�digo problem�ticos. Para a primeira tarefa, desenvolvedores buscam se apoiar em boas pr�ticas e \textit{design patterns} j� estabelecidos \cite{alur2003core, gamma1994design, rozanski2012software}. Para a segunda tarefa, � comum utilizar estrat�gias de detec��o de maus cheiros de c�digo \cite{FinavaroAniche2016}, que apontam trechos de c�digos que podem se beneficiar de refatora��o, ou seja, melhorar o c�digo sem alterar o comportamento \cite{RefactoringFowler1999}. Apesar de j� existir um cat�logo extenso de maus cheiros, eles n�o levam em considera��o a natureza do projeto e suas particularidades. 

Conforme relatado por Hecht \cite{Hecht2015} com rela��o a projetos Android, ``\textit{antipatterns} espec�ficos � plataforma Android s�o mais comuns e ocorrem mais frequentemente do que \textit{antipaterns} OO (Orientados a Objetos)'' (tradu��o livre). Vale lembrar que al�m de c�digo Java, grande parte de um projeto Android � constitu�do por arquivos \texttt{XML}. Estes s�o os \emph{recursos da aplica��o} (do ingl�s \textit{Application Resources}) e ficam localizados no diret�rio \texttt{res} do projeto. S�o respons�veis por apresentar algo ao usu�rio como, uma tela, uma imagem, uma tradu��o e assim por diante. No in�cio do projeto, os recursos costumam ser poucos e pequenos. Conforme o projeto evolui, a quantidade e complexidade dos recursos tende a aumentar, trazendo problemas para encontr�-los, reaproveit�-los e entend�-los. Enquanto esses problemas j� est�o bem resolvidos em projetos orientados a objetos, ainda n�o � trivial encontrar uma forma sistem�tica de identific�-los em recursos de projetos Android. 

Outra caracter�stica � com rela��o � \textsc{Activities}, que s�o classes espec�ficas da plataforma Android respons�veis pela apresenta��o e intera��es do usu�rio com a tela \cite{AndroidActivities2016}. \textsc{Activities} tamb�m possuem muitas responsabilidades \cite{MobileSmells:13}, est�o vinculadas a um \textsc{Layout} que representa uma interface com o usu�rio e normalmente precisam de acesso a classes do modelo da aplica��o. Analogamente ao padr�o MVC, \textsc{Activities} fazem os pap�is de \textsc{View} e \textsc{Controller} simultaneamente. Isto posto, � razo�vel considerar que o mau cheiro \textit{God Class} \cite{riel1996object} � aplic�vel nesse caso, no entanto, conforme bem pontuado por Aniche et al. \cite{FinavaroAniche2016} ``\emph{enquanto [\textit{God Class}] se encaixa bem em qualquer sistema orientado a objetos, ele n�o leva em considera��o as particularidades arquiteturais da aplica��o ou o papel desempenhado por uma determinada classe}.'' (tradu��o livre). 

Na pr�tica, desenvolvedores Android percebem estes problemas frequentemente. Muitos deles j� se utilizam de pr�ticas para solucion�-los, conforme relatado por Reimann et al. \cite{ReimannBrylski2013} ``\emph{o problema no desenvolvimento m�vel � que desenvolvedores est�o cientes sobre maus cheiros apenas indiretamente porque estas defini��es [dos maus cheiros] s�o informais (boas pr�ticas, relat�rios de problemas, f�runs de discuss�es, etc.) e recursos onde encontr�-los est�o distribu�dos pela internet}'' (tradu��o livre). Ou seja, n�o � encontrado atualmente um cat�logo �nico de boas e m�s pr�ticas, tornando dif�cil a detec��o e sugest�o de refatora��es apropriadas �s particularidades da plataforma. 

Pesquisas sobre Android ainda s�o poucas. Nas principais confer�ncias de manuten��o de software, dentre 2008 a 2015, 5 artigos foram sobre maus cheiros Android, dentro de um total de 52 artigos sobre o assunto \cite{Mannan_Dig_Ahmed_Jensen_Abdullah_Almurshed}. A aus�ncia de um cat�logo de maus cheiros Android resulta em (i) uma car�ncia de conhecimento sobre boas e m�s pr�ticas a ser compartilhado entre praticantes da plataforma, (ii) indisponibilidade de uma ferramenta de detec��o de maus cheiros de forma a alertar automaticamente os desenvolvedores da exist�ncia dos mesmos e (iii) aus�ncia de estudo emp�rico sobre o impacto dessas m�s pr�ticas na manutenibilidade do c�digo de projetos Android. \\ \ \\



\section{Quest�es de Pesquisa}
Esta disserta��o tem por objetivo investigar maus cheiros espec�ficos � camada de apresenta��o de projetos Android. Desta forma, trabalhamos a seguinte quest�o de pesquisa:

\begin{center}
\textbf{Existem Maus Cheiros espec�ficos � Camada de Apresenta��o Android?}
\end{center}

Para isso, exploramos as seguintes quest�es: \\

\textbf{Q1: O que desenvolvedores consideram boas e m�s pr�ticas com rela��o � Camada de Apresenta��o em projetos Android?}

Nesta quest�o, n�s investigamos a exist�ncia de maus cheiros em elementos da camada de apresenta��o Android como \textsc{Activities} e \textsc{Adapters}. Para responder a esta pergunta aplicamos question�rio e realizamos entrevistas com desenvolvedores especialistas em Android. Tamb�m coletamos postagens em f�runs e blogs t�cnicos sobre Android. \\


\textbf{Q2: Qual a rela��o entre os maus cheiros propostos e a tend�ncia a mudan�as e defeitos no c�digo?}

Estudos pr�vios mostram que maus cheiros tradicionais (e.g., \textit{Blob Classes}) podem impactar na tend�ncia a mudan�as em classes do projeto \cite{FinavaroAniche2016}. Desta forma, esta quest�o pretende, por meio de um experimento com desenvolvedores Android, analisar o impacto dos maus cheiros propostos na tend�ncia a mudan�as e defeitos em projetos Android. \\


\textbf{Q3: Desenvolvedores Android percebem os c�digos afetados pelos maus cheiros propostos como problem�ticos?}

Com esta quest�o complementamos com dados qualitativos as an�lises quantitativas realizadas no contexto de Q2. Desta forma, investigamos se c�digos afetados pelos maus cheiros definidos para a camada de apresenta��o Android s�o percebidos como problem�ticos por desenvolvedores. \\ \

Fizemos uso de diferentes m�todos de pesquisa durante esta disserta��o. Desta forma, cada m�todo usado � abordado no cap�tulo respectivo � quest�o. Todos os cap�tulos exigem do leitor conhecimento pr�vio sobre Android, Maus Cheiros de C�digo e M�tricas de C�digo. Apresentamos uma breve introdu��o a esses tr�s assuntos no cap�tulo \ref{cap:background}.

\section{Contribui��es}

As principais contribui��es desta disserta��o, na ordem em que aparecem, s�o:

\begin{enumerate}
	\item A defini��o do termo \textbf{Camada de Apresenta��o Android}. Com embasamento te�rico sobre a origem de interfaces gr�ficas e na documenta��o oficial do Android provemos uma defini��o sobre quais elementos comp�em a camada de apresenta��o Android. 

	\item Um cat�logo validado de maus cheiros da camada de apresenta��o Android. Os maus cheiros foram definidos com a participa��o de mais de 50 desenvolvedores em question�rios e entrevistas.

	\item Um estudo quantitativo sobre a tend�ncia a mudan�as e defeitos dos maus cheiros propostos. Realizaremos um experimento com desenvolvedores Android de modo a coletar quantitativamente se classes afetadas pelos maus cheiros possuem uma maior tend�ncia a mudan�as e introdu��o de defeitos.

	\item Um estudo sobre a percep��o de desenvolvedores sobre os maus cheiros propostos. Realizaremos um experimento com desenvolvedores Android de modo a identificar se classes afetadas pelos maus cheiros s�o percebidas como problem�ticas por desenvolvedores Android.
\end{enumerate}

\section{Organiza��o da Disserta��o}

O restante desta disserta��o est� organizada da seguinte forma:

\begin{itemize}
	\item \textbf{Cap�tulo 2} Fundamenta��o Conceitual
	
	Neste cap�tulo � passado ao leitor informa��es b�sicas relevantes para o entedimento do trabalho. Os assuntos aprofundados aqui s�o: Qualidade de C�digo, Maus Cheiros e Android.

	\item \textbf{Cap�tulo 3} Trabalhos Relacionados

	Neste cap�tulo pretende-se apresentar estudos relevantes j� feitos em torno do tema de maus cheiros Android e em que esta disserta��o se diferencia deles.

	\item \textbf{Cap�tulo 4} Camada de Apresenta��o Android

	Esta pesquisa limita-se em mapear boas e m�s pr�ticas apenas na camada de apresenta��o de aplicativos Android. Neste cap�tulo pretende-se explanar para o leitor o que � considerado como camada de apresenta��o Android.

	\item \textbf{Cap�tulo 5} Proposta de Disserta��o

	Neste cap�tulo apresentamos a proposta da disserta��o e o cronograma de atividades.

	\item \textbf{Cap�tulo 6} Boas e M�s Pr�ticas na Camada de Apresenta��o

	Neste cap�tulo respondemos a Q1. � apresentada a motiva��o da quest�o, os m�todos de pesquisa utilizados e o cat�logo resultante de maus cheiros.

	\item \textbf{Cap�tulo 7} Impacto na Tend�ncia a Mudan�as e Defeitos

	Neste cap�tulo respondemos a Q2. � apresentada a motiva��o da quest�o, explicamos o experimento conduzido e os resultados obtidos.

	\item \textbf{Cap�tulo 8} Percep��o dos Desenvolvedores

	Neste cap�tulo respondemos a Q3. � apresentado a motiva��o da quest�o, explicamos o experimento conduzido e os resultados obtidos.

	\item \textbf{Cap�tulo 9} Conclus�o

	Neste cap�tulo s�o apresentadas as conclus�es do trabalho, bem como as suas limita��es e sugest�es de trabalhos futuros.
\end{itemize}

 
% -*- root: dissertacao.tex -*-
%%%%%%%%%%%%%%%%%%%%%%%%%%%%%%%%%%%%%%%%%%%%%%%%%%%%%%%%%%%%%%%%%%%%%%%
\setlength{\parindent}{20pt}
\setlength{\textheight}{22cm}
\setlength{\parskip}{0.2cm}
\linespread{1.2} % Para aumentar o espa�amento entre as linhas
%%%%%%%%%%%%%%%%%%%%%%%%%%%%%%%%%%%%%%%%%%%%%%%%%%%%%%%%%%%%%%%%%%%%%%%

\chapter{Fundamenta��o Te�rica}

Para a compreens�o deste trabalho � importante ter claro a defini��o de 3 itens, s�o eles: Qualidade de C�digo, \textit{Code Smells} e \textit{Android}.

\section{M�tricas de C�digo}


\section{Code Smells}

Mau cheiro de c�digo � uma indica��o superficial que usualmente corresponde a um problema mais profundo em um software. Por si s� um \textit{code smell}, seu termo em ingl�s, n�o � algo ruim, ocorre que frequentemente ele indica um problema mas n�o necess�riamente � o problema em si \cite{CodeSmell:06}. O termo em ingl�s \textit{code smell} foi cunhado pela primeira vez por Kent Beck enquanto ajudava Martin Fowler com o seu livro Refactoring \cite{Refactoring:99} \cite{CodeSmell:06}.

\textit{Code Smells} s�o padr�es de c�digo que est�o associados com um design ruim e m�s pr�ticas de programa��o. Diferentemente de erros de c�digo eles n�o resultam em comportamentos erroneos. \textit{Code Smells} apontam para �reas na aplica��o que podem se beneficiar de refatora��es. \cite{MobileSmells:13}. Refatora��o � definido por ``uma t�cnica para reestrutura��o de um c�digo existente, alterando sua estrutura interna sem alterar seu comportamento externo'' \cite{Refactoring:99}.

Escolher n�o resolver \textit{code smells} pela refatora��o n�o resultar� na aplica��o falhar mas ir� aumentar a dificuldade de mant�-la. Logo, a refatora��o ajuda a melhorar a manutenabilidade de uma aplica��o \cite{MobileSmells:13}. Uma vez que os custos com manuten��o s�o a maior parte dos custos envolvidos no ciclo de desenvolvimento de software \cite{RefactoringAndImprovements:10}, aumentar a manutenabilidade atrav�s de refatora��o ir� reduzir os custos de um software no longo prazo. 


\section{Android}

\subsection{Arquitetura da Plataforma}

Android � um sistema operacional de c�digo aberto, baseado no kernel do Linux criado para um amplo conjunto de dispositivos. Para prover acesso aos recursos espec�ficos dos dispositivos como c�mera ou \textit{bluetooth}, o Android possui uma camada de abstra��o de \textit{hardware} (HAL do ingl�s \textit{Hardware Abstraction Layer}) exposto aos desenvolvedores atrav�s de um arcabou�o de interfaces de programa��o de aplicativos (APIs do ingl�s \textit{Applications Programming Interface}) Java. Estes e outros elementos explicados a seguir podem ser visualizados na figura \ref{fig:AndroidPlatform} \cite{AndroidPlatformArchitecture}.

\begin{figure}[!htb]
	\centering
	\includegraphics[width=0.7\textwidth]{android-architecture.png}
	\caption{Arquitetura do sistema operacional Android.}
	\label{fig:AndroidPlatform}
\end{figure}

Cada aplicativo � executado em um novo processo de sistema que cont�m sua pr�pria inst�ncia do ambiente de execu��o Android. A partir da vers�o 5 (API n�vel 21), o ambiente de execu��o padr�o � o Android Runtime (ART), antes desta vers�o era a Dalvik. ART foi escrita para executar multiplas inst�ncias de m�quina virtual em dispositivos com pouca mem�ria. Suas funcionalidades incluem duas forma de compila��o, a frente do tempo (AOT do ingl�s \textit{Ahead-of-time}) e apenas no momento (JIT do ingl�s \textit{Just-in-time}), coletor de lixo, ferramentas de depura��o, relat�rio de diagn�sticos de erros e exce��es.

Muitos do componentes e servi�os b�sicos do Android, como ART e HAL, foram criados a partir de c�digo nativo que depende de bibliotecas nativas escritas em C e C++. A plataforma Android prov� arcabou�os de APIs Java para exp�r as funcionalidade de algumas destas bibliotecas nativas para os aplicativos. Por exemplo, OpenGL ES pode ser acessado atrav�s do arcabou�o Android Java OpenGL API, de forma a adicionar suporte ao desenho e manipula��o de gr�ficos 2D e 3D no aplicativo.

Todo o conjunto de funcionalidades da plataforma Android est� dispon�vel para os aplicativos atrav�s de APIs Java. Estas APIs comp�em os elementos b�sicos para a constru��o de aplicativos Android. Dentre eles, os mais relevantes para esta disserta��o s�o:

\begin{itemize}
	\item Um rico e extens�vel \textbf{Sistema de Visualiza��o} para a contru��o das interfaces com o usu�rio, tamb�m chamadas de \textit{layouts}, do aplicativo. Incluindo listas, grades, caixas de textos, bot�es, dentre outros.

	\item Um \textbf{Gerenciador de Recursos}, provendo acesso aos recursos ``n�o-java'' como textos, elementos gr�ficos, arquivos de \textit{layout}.

	\item Um \textbf{Gerenciador de Activity} que gerencia o ciclo de vida dos aplicativos e prov� uma navega��o comum.
\end{itemize}

O Android j� vem com um conjunto de aplicativos b�sicos como por exemplo, para envio e recebimento de SMS, calend�rio, navegador, contatos e outros. Estes aplicativos vindos com a plataforma n�o possuem nenhum diferencial com rela��o aos aplicativos de terceiros. Todo aplicativo tem acesso ao mesmo arcabou�o de APIs do Android, seja ele aplicativo da plataforma ou de terceiro. Desta forma, um aplicativo de terceiro pode se tornar o aplicativo padr�o para navegar na internet, receber e enviar SMS e assim por diante.

Aplicativos da plataforma provem capacidades b�sicas que aplicativos de terceiros podem reutilizar. Por exemplo, se um aplicativo de terceiro quer possibilitar o envio de SMS, o mesmo pode redirecionar esta funcionalidade de forma a abrir o aplicativo de SMS j� existente, ao inv�s de implementar por si s�.

\subsection{Aplicativos Android}

Aplicativos Android s�o escritos na linguagem de programa��o Java. O Kit para Desenvolvimento de Software (SDK do ingl�s \textit{Software Development Kit}) Android compila o c�digo, junto com qualquer arquivo de recursos ou dados, em um arquivo Android Package (APK). Um APK, arquivo com extens�o \texttt{.apk}, � usado por dispositivos para a instala��o de um aplicativo \cite{AndroidFundamentals}.

Componentes Android s�o os elementos base para a constru��o de aplicativos Android. Cada componente � um diferente ponto atrav�s do qual o sistema pode acionar o aplicativo. Nem todos os componente s�o pontos de entrada para o usu�rio e alguns s�o dependentes entre si, mas cada qual existe de forma aut�noma e desempenha um papel espec�fico. 

Existem quato tipos diferentes de componentes Android. Cada tipo serve um prop�sito distinto e tem diferentes ciclos de vida que definem como o componente � criado e destru�do. O quatro componentes s�o:

\begin{itemize}

	\item \textbf{Activities}

	Uma \textit{activity} representa uma tela com uma interface de usu�rio. Por exemplo, um aplicativo de email pode ter uma \textit{activity} para mostrar a lista de emails, outra para redigir um email, outra para ler emails e assim por diante. Embora \textit{activities} trabalhem juntas de forma a criar uma experi�ncia de usu�rio (UX do ingl�s \textit{User Experience}) coesa no aplicativo de emails, cada uma � independente da outra. Desta forma, um aplicativo diferente poderia iniciar qualquer uma destas \textit{activities} (se o aplicativo de emails permitir). Por exemplo, a \textit{activity} de redigir email no aplicativo de emails, poderia solicitar o aplicativo c�mera, de forma a permitir o compartilhamento de alguma foto. Uma \textit{activity} � implementada como uma subclasse de \texttt{Activity}.  

	\item \textbf{Services}

	Um servi�o � um componente que � executado em plano de fundo para processar opera��es de longa dura��o ou processar opera��es remotas. Um servi�o n�o prov� uma interface com o usu�rio. Por exemplo, um servic� pode tocar uma m�sica em plano de fundo enquanto o usu�rio est� usando um aplicativo diferente, ou ele pode buscar dados em um servidor remoto atrav�s da internet sem bloquear as intera��es do usu�rio com a activity. Outros componente, como uma activity, pode iniciar um servi�o e deix�-lo executar em plano de fundo, durante sua execu��o � poss�vel interagir com ele. Um servi�o � implementado como uma subclasse de \texttt{Service}.

	\item \textbf{Content Providers}

	Um provedor de conte�do gerencia um conjunto compartilhado de dados do aplicativo. � poss�vel armazenar dados em arquivos de sistema, um banco de dados SQLite, num servidor remoto ou qualquer outro local de armazenamento que o aplicativo possa acessar. Atrav�s de provedores de conte�do, outros aplicativos podem consultar ou modificar os dados (se o provedor de conte�do permitir). Por exemplo, a plataforma Android disponibiliza um provedor de conte�do que gerencia as informa��es dos contatos dos usu�rios. Desta forma, qualquer aplicativo, com as devidas permiss�es, pode consultar parte do provedor de conte�do (como \texttt{ContactsContract.Data}) para ler e escrever informa��es sobre um contato espec�fico. Um provedor de conte�do � implementado como uma subclasse de \texttt{ContentProvider}.

	\item \textbf{Broadcast Receivers}

	Um recebedor de mensagens � um componente que responde a mensagens enviadas pelo sistema. Muitas destas mensagens s�o originadas da plataforma Android, por exemplo o desligamento da tela, baixo n�vel de bateria e assim por diante. Aplicativos de terceiros tamb�m podem enviar mensagens, por exemplo, informando que alguma opera��o foi conclu�da, com um email que concluiu seu envio. No entanto, recebedor de mensagens n�o possuem interface de usu�rio, eles podem criar notifica��es para alertarem o usu�rio que algo ocorreu. Um recebedor de mensagens � implementado como uma subclasse de \texttt{BroadcastReceiver}.

\end{itemize}

Antes de a plataforma Android poder iniciar qualquer componente do aplicativo, ela precisa saber que eles existem. Isso � feito atrav�s da leitura do arquivo \texttt{AndroidManifest.xml} do aplicativo (arquivo de manifesto). O aplicativo deve declarar todos seus componentes neste arquivo, que deve estar no diret�rio raiz do projeto.

O arquivo de manifesto pode conter muitas outras informa��es al�m das declara��es dos componentes do projeto, por exemeplo:

\begin{itemize}
	\item Identificar qualquer permiss�o de usu�rio requerida pelo aplicativo, como acesso a internet, acesso a informa��es de contatos do usu�rio e assim por diante.

	\item Declarar o n�vel m�nimo do Android requerido para o aplicativo, baseado em quais APIs s�o usadas pelo aplicativo.

	\item Declarar quais funcionalidades de sistema ou \textit{hardware} s�o usadas ou requeridas pelo aplicativo, por exemplo c�mera, \textit{bluetooth} e assim por diante.

	\item Declarar outras APIs que s�o necess�rias para uso do aplicativo (al�m do arcabou�o de APIs do Android), como a biblioteca do Google Maps.
\end{itemize}

O arquivo de manifesto � um arquivo XML. Os elementos usados nele s�o defidos pelo Android, por exemplo, um arquivo de manifesto pode declarar uma atividade conforme a seguir. \\

\begin{lstlisting}[language=XML, caption=Arquivo \texttt{AndroidManifest.xml}]
<?xml version="1.0" encoding="utf-8"?>
<manifest ... >
    <application android:icon="@drawable/app_icon.png" ... >
        <activity android:name="com.example.project.ExampleActivity"
                  android:label="@string/example_label" ... >
        </activity>
        ...
    </application>
</manifest>	
\end{lstlisting}

No elemento \texttt{<application>} o atributo \texttt{android:icon} aponta para o �cone, que � um recurso, que identifica o aplicativo.

No elemento \texttt{<activity>}, o atributo \texttt{android:name} especifica o nome da classe completamente qualificado de uma subclasse de \texttt{Activity} e o atributo \texttt{android:label} especifica um texto para ser usado como t�tulo da atividade.

Deve-se declarar todos os componentes do aplicativo usando os seguintes elementos:
\begin{itemize}
	\item \texttt{<activity>} elemento para atividades.
	\item \texttt{<service>} elemento para servi�os.
	\item \texttt{<receiver>} elemento para recebedor de mensagens.
	\item \texttt{<provider>} elemento para provedores de conte�do.
\end{itemize}

Um aplicativo Android � composto por outros arquivos al�m de c�digo Java, ele requer \emph{recursos} como imagens, arquivos de �udio, e qualquer recurso relativo a apresenta��o visual do aplicativo \cite{AndroidFundamentals}. Tamb�m � poss�vel definir menus, estilos, cores e \textit{layout} das atividades.

For example, you should define animations, menus, styles, colors, and the layout of activity user interfaces with XML files. Using app resources makes it easy to update various characteristics of your app without modifying code and�by providing sets of alternative resources�enables you to optimize your app for a variety of device configurations (such as different languages and screen sizes).

 
% -*- root: dissertacao.tex -*-
%%%%%%%%%%%%%%%%%%%%%%%%%%%%%%%%%%%%%%%%%%%%%%%%%%%%%%%%%%%%%%%%%%%%%%%
\setlength{\parindent}{20pt}
\setlength{\textheight}{22cm}
\setlength{\parskip}{0.2cm}
\linespread{1.2} % Para aumentar o espa�amento entre as linhas
%%%%%%%%%%%%%%%%%%%%%%%%%%%%%%%%%%%%%%%%%%%%%%%%%%%%%%%%%%%%%%%%%%%%%%%

% Identificar o contexto e a motiva��o para o trabalho, especificar o problema, discutir trabalhos relacionados (principalmente as suas limita��es), definir as contribui��es, enfatizar os resultados principais e a organiza��o do texto.

\chapter{Trabalhos Relacionados}
\label{cap:related}

Agrupamos os trabalhos relacionados em 3 se��es: a Se��o 3.1 trata de estudos recentes realizados sobre Android, a Se��o 3.2 trata de estudos que analisam cheiros de c�digo espec�ficos � alguma tecnologia ou plataforma e a Se��o 3.3, estudos sobre cheiros de c�digo espec�ficos � plataforma Android.

\section{Pesquisas Focadas na Plataforma Android}

Diversas pesquisas t�m sido realizadas em torno da plataforma Android. � poss�vel encontrar artigos sobre temas variados, como por exemplo seguran�a \cite{Y, X, D, E, F, G, H}, an�lise est�tica de c�digo \cite{A, B, C, X, Z} e autentica��o de usu�rio \cite{P,Q,R,S}. Neste se��o passamos por alguns desses e outros de modo a apresentar uma vis�o geral das pesquisas realizadas sobre Android. 

% usabilidade

Adrienne et al. \cite{AdriennePermission} realizou um estudo de usabilidade com usu�rios Android para entender a efetividade do sistema de permiss�es de usu�rios. Quando um usu�rio instala uma aplica��o tem a oportunidade de rever as permiss�es solicitadas pelo aplicativo. O estudo concluiu que 17\% dos usu�rios prestam aten��o as permiss�es apresentadas durante a instala��o. O estudo conclui com recomenda��es para melhorar o entendimento e aten��o dos usu�rios com rela��o as permiss�es solicitadas. 
% ----------------------------------------------------------------- 

% Seguran�a & An�lise Est�tica
O sistema operacional Android al�m de possuir uma API aberta, tem um rico sistema de mensagens entre aplica��es. Isso reduz o trabalho de desenvolvimento, facilitando a reutiliza��o de componentes \cite{Y}. Entretanto o Android confere uma responsabilidade significativa aos desenvolvedores de aplicativos com rela��o ao risco de problemas de seguran�a como afirma Kavitha K et al. \cite{X}. 

Infelizmente a comunica��o entre aplica��es pode ser detectada, capturada ou at� mesmo substitu�da, comprometendo a privacidade e seguran�a do usu�rio como a Erika Chin et al. \cite{Y} afirma. 

Alguns autores tem explorando os riscos de problemas de seguran�a envolvendo Android \cite{Y, X, D, E, F, G, H}. No artigo Analyzing Inter-Application Communication in Android os autores examinam itera��es entre aplica��es Android e identificam os riscos de seguran�a nos componentes. Enquanto no artigo Exploring the Malicious Android Applications and Reducing Risk using Static Analysis os autores investigam os problemas de seguran�a envolvendo o controle de permiss�o. 

Uma das contribui��es do trabalho de Erika Chin et al. \cite{Y} � uma ferramenta para detec��o de vulnerabilidades na comunica��o entre aplica��es chamada ComDroid. A ferramenta ComDroid baseia-se no DEX do aplicativo. Isto permite que terceiros ou revisores poderem utilizar a ferramenta para avaliar aplicativos mesmo quando n�o tem dispon�vel o c�digo fonte. O funcionamento da ferramenta ComDroid � baseado na an�lise est�tica da sa�da do Dedexer \cite{P}. A an�lise est�tica tem sido comumente utilizada para a detec��o de bugs \cite{A, B, C}. 

Kavitha K \cite{X} tamb�m utiliza an�lise estat�stica em conjunto com an�lise din�mica em um sistema de detec��o de malware com base no controle de permiss�o do Android e os passos necess�rios para mitigar o acesso a permiss�es indesejadas. 

No artigo Automated Static Code Analysis for Classifying Android Applications Using Machine Learning \cite{Z} os autores utilizam an�lise est�tica para  classificar aplica��es Android  em dois tipos: ferramentas e jogos. Do mesmo modo que o trabalho de Erika Chin et al. \cite{Y} os autores Asaf Shabtai et al. \cite{Z} tamb�m se baseiam no DEX, entretanto, mesmo utilizando aprendizado de m�quina em conjunto com an�lise est�tica alguns aplicativos n�o conseguiram ser classificados corretamente.

Thomas Bl�sing et al \cite{K} no artigo An Android Application Sandbox System for Suspicious Software Detection propoem um sandbox para an�lises de aplica��es Android denominado AASandbox. O AASandbox pode ser implantado na nuvem e � capaz de executar an�lises est�ticas e din�micas em programas Android com objetivo de detectar automaticamente aplicativos suspeitos. A an�lise est�tica do software tem foco em padr�es maliciosos utilizando servi�os na nuvem utilizando um emulador Android em ambiente isolado, fornecendo uma detec��o r�pida e distribu�da de software suspeito. Al�m disso, o AASandbox pode ser utilizado para melhorar a efici�ncia dos antiv�rus dispon�veis para o sistema operacional Android.


% Teste de Aplicativos Android
Os usu�rios confiam cada vez mais em aplica��es m�veis para necessidades computacionais. Android � uma plataforma m�vel muito popular, desta maneira a confiabilidade de aplicativos Android est� se tornando cada vez mais importante \cite{J, L}. De acordo com Wasserman \cite{N}, o desafio de engenharia de software com desenvolvimento de aplica��es m�veis � o de encontrar solu��es eficazes para alcan�ar qualidades e definir t�cnicas e ferramentas adequadas para suportar seus testes. Testar aplicativos do Android � uma atividade desafiadora, com v�rios problemas abertos, problemas espec�ficos e perguntas em aberto principalmente em rela��o a GUI \cite{J, L, M}.


No artigo Automating GUI Testing for Android Applications os autores apresentam uma abordagem para automatizar o processo de teste para aplicativos Android com foco em bugs de interface, como erros de ACTIVITY e EVENT. Primeiramente os autores realizam um estudo de bugs para entender a natureza e a frequ�ncia de bugs que afetam aplicativos Android. Em seguida Cuixiong Hu et al \cite{J} apresentam t�cnicas de detec��o de erros GUI utilizando gera��o autom�tica de casos de teste, utilizando eventos aleat�rios da aplica��o e instrumenta��o da m�quina virtual, produzindo arquivos de log que ser�o utilizados na an�lise p�s-execu��o. As t�cnicas apresentadas mostraram-se eficazes para erros de ACTIVITY, EVENT e TYPE ERRORS.


Atif M. Memon no artigo Using GUI Ripping for Automated Testing of Android Applications \cite{M}  apresenta o AndroidRipper uma t�cnica automatizada que testa aplicativos Android atrav�s da GUI. AndroidRipper � baseado em um explorador autom�tico da GUI do aplicativo com o objetivo de exercitar o aplicativo de forma estruturada. Uma das contribui��es do artigo \cite{M} � a disponibiliza��o do aplicativo  AndroidRipper com c�digo aberto. Os resultados apresentados demonstra a capacidade de detectar falhas graves, previamente desconhecidas no c�digo, e que a explora��o estruturada supera a abordagem aleat�ria.

% -----------------

Dispositivos m�veis tornaram-se uma parte importante na vida das pessoas \cite{P, S}. A identifica��o de dispositivos � de grande import�ncia para a autentica��o segura de usu�rios em redes m�veis e tem atra�do a aten��o de muitos pesquisadores \cite{P, S, T}. 

Yildirim et a. \cite{S} prop�e o uso do leitor biom�trico somado a identidade de equipamento m�vel internacional (IMEI) para a gera��o de uma senha �nica e expir�vel para a autentica��o em sistemas web. Alguns fabricantes de dispositivos m�veis permitem que os desenvolvedores usem os recursos de seguran�a do dispositivo em seus aplicativos por meio do kit de desenvolvimento do dispositivo (SDK) do dispositivo.

Wu et al. \cite{P} prop�e que os dados obtidos a partir de um smartphone podem ser usados para identificar o usu�rio do smartphone, sem a necessidade de alguma a��o pelo usu�rio. O autor afirma que M�todos tradicionais utilizam identificadores expl�citos como o IMEI, a identidade de assinante m�vel internacional (IMSI) ou o Android ID. No entanto, alguns identificadores expl�citos n�o s�o confi�veis e podem ser facilmente adulterados ou forjados, al�m de que, para obt�-los, � necess�rio solicitar permiss�o do usu�rio o que pode causar em abuso de permiss�es sens�veis e vazamento da privacidade \cite{Q, R}.

Para resolver esses problemas, Wu et al. \cite{P} prop�e tr�s algoritmos de identifica��o de dispositivos baseado em identificadores impl�citos que podem ser obtidos sem solicitar qualquer permiss�o \cite{P}. Os identificadores impl�citos s�o compostos por dados obtidos (i) da camada f�sica (hardware) como espa�o em disco, (ii) da camada de aplica��o (sistema operacional) como vers�o do Android e (iii)  da camada do usu�rio (configura��es) como time-zone. 


% ----------------------------------------------------------------- 



\section{Cheiros de C�digo Espec�ficos}

Diversos pesquisadores propuseram cheiros de c�digo e pr�ticas recomendadas para tecnologias ou plataformas espec�ficas, como frameworks Java \cite{AA, FinavaroAniche2016}, linguagem web como Cascading Style Sheets (CSS) \cite{CSSCodeSmell} e Javascript \cite{BB} e outros.

Chen et al. \cite{AA} afirma que frameworks Object-Relational Mapping (ORM) s�o amplamente utilizado na ind�stria. No entanto, os desenvolvedores geralmente escrevem c�digo ORM sem considerar o impacto desse c�digo no desempenho de banco de dados, levando a causar transa��es com "timeout" ou travamentos em sistemas em larga escala. Chen et al. \cite{AA} soluciona este problema com implementa��o de um framework automatizado e sistem�tico para detectar e priorizar anti-patterns de desempenho para aplica��es desenvolvidas usando ORM. Estudos de caso mostraram que o framework pode detectar centenas ou milhares de inst�ncias de anti-patterns de desempenho ao mesmo tempo que prioriza efetivamente a corre��o dessas inst�ncias \cite{AA}. Foi descoberto que a corre��o dessas inst�ncias de anti-patterns de desempenho pode melhorar o tempo de resposta dos sistemas em at� 98\% (e, em m�dia, 35\%). Al�m do framework que � extens�vel podendo agregar outros anti-patterns, Chen et al. \cite{AA} contribui com o mapeamento de 2 anti-patterns espec�ficos a frameworks ORM.

Aniche et al. \cite{FinavaroAniche2016} tamb�m investigou cheiros de c�digo relacionado a um framework. Segundo o autor, para escrever c�digo f�cil de ser mantido e evolu�do, e detectar peda�os de c�digo problem�ticos, desenvolvedores fazem uso de m�tricas de c�digo e estrat�gias de detec��o de maus cheiros de c�digo. No entanto, m�tricas de c�digo e estrat�gias de detec��o de maus cheiros de c�digo n�o levam em conta a arquitetura do software em an�lise o que significa que todas classes s�o avaliadas como se umas fossem iguais �s outras. Aniche et al. \cite{FinavaroAniche2016} afirma que cada papel arquitetural possui resposabilidades diferentes o que resulta em distribui��es diferentes de valores de m�trica de c�digo. Mostra ainda que classes que cumprem um papel arquitetural espec�fico, como por exemplo \textsc{Controllers}, tamb�m cont�m maus cheiros de c�digo espec�ficos. Uma das contribui��es de Aniche et al. � um cat�logo com 6 cheiros de c�digos espec�ficos ao framework Spring MVC mapeados e validados.

CSS � amplamente utilizado nas aplica��es web de hoje para separar a sem�ntica de apresenta��o do conte�do HTML \cite{CSSCodeSmell}. De acordo como Gharachorlu \cite{CSSCodeSmell} apesar da simplicidade de s�ntaxe do CSS, as caracter�sticas espec�ficas da linguagem tornam a cria��o e manuten��o de CSS uma tarefa desafiadora. Foi realizando um estudo emp�rico de larga escala em 500 sites, 5060 arquivos no total, que consistem de mais de 10 milh�es de linhas de c�digo CSS. Segundo o autor, os resultados indicaram que o CSS de hoje sofre significativamente de padr�es inadequados e est� longe de ser um c�digo bem escrito. Porfim Gharachorlu \cite{CSSCodeSmell} prop�e o primeiro modelo de qualidade de c�digo CSS derivado de uma grande amostra de aprendizagem de modo a ajudar desenvolvedores a obter uma estimativa do n�mero total de cheiros de c�digo em seu c�digo CSS. Sua principal contrinbui��o foi oito novos cheiros de c�digo CSS detectados com o uso da ferramenta CSSNose, tamb�m implementada e disponibilizada pelo autor.

Javascript � uma flex�vel linguagem de script para o desenvolvimento de aplica��es Web 2.0 \cite{BB}. Fard e Ali \cite{BB} afirmam que devido � essa flexibilidade, o JavaScript � uma linguagem particularmente desafiadora para escrevere manter c�digo. Os desafios s�o m�ltiplos: Primeiro, � uma linguagem interpretada, o que significa que normalmente n�o h� compilador no ciclo de desenvolvimento que ajudaria os desenvolvedores a detectar c�digo incorreto ou n�o otimizado. Segundo, tem uma natureza din�mica, fracamente tipificada, ass�ncrona. Terceiro, ele suporta recursos intrincados, como prototypes \cite{CC}, fun��es de primeira classe e "closures" \cite{DD}. E finalmente, ele interage com o DOM atrav�s de um mecanismo complexo baseado em eventos \cite{EE}. Os autores prop�em um conjunto de 13 cheiros de c�digo JavaScript, sendo 7 cheiros de c�digos bem conhecidos adaptados para o JavaScript e 6 tipos espec�ficos de c�digos de JavaScript devidos do trabalho. Tamb�m � apresentada uma t�cnica automatizada, chamada JSNOSE, para detectar esses cheiros de c�digo.

Ab�lio et al \cite{FF} descreve no artigo um estudo explorat�rio sobre maus cheiros de c�digo em um ambiente que n�o utiliza engenharia de software convencional. A abordagem das Linhas de Produtos de Software (Software Product Lines, SPL) centra-se no uso de t�cnicas de engenharia que permitem criar um grupo de sistemas de software similares a partir de um conjunto de especifica��es de software comuns a todos esses sistemas. Diferenciando  da engenharia de software convencional principalmente pela presen�a de varia��o em alguns ou at� todos os requisitos de software \cite{GG, II, HH}. Programa��o Orientada a Recursos (do ingl�s \textit{Feature-Oriented Programming}, FOP) � um paradigma para a modulariza��o de software em que as caracter�sticas s�o as principais abstra��es \cite{JJ}. Os recursos podem ser realizados em artefatos separados usando abordagens de composi��o com FOP e Programa��o Orientada a Aspecto (do ingl�s, \textit{Aspect-Oriented Programming}, AOP) \cite{LL}.

No entanto, a Programa��o Orientada a Recursos � uma t�cnica espec�fica para lidar com a modulariza��o de recursos no SPL \cite{FF}. Uma das linguagens FOP mais populares � a AHEAD e ainda faltam estudos sistem�ticos sobre a categoriza��o e detec��o de cheiros de c�digo em SPL baseado em AHEAD \cite{FF}. Para preencher essa lacuna, Ab�io et al. \cite{FF} estende as defini��es de tr�s maus cheiros de c�digo tradicionais, \textit{God Method}, \textit{God Class}, e \textit{Shotgun Surgery}, para levar em conta as abstra��es de FOP. Propondo novas m�tricas FOP para quantificar caracter�sticas espec�ficas de abordagens com o AHEAD e definindo estrat�gias para identificar os maus cheiros de c�digo investigados.


\section{Cheiros de C�digo Espec�ficos ao Android}

% Em outros trabalhos sobre cheiros de c�digo Android, foram identificados cheiros de c�digo relacionados a seguran�a, consumo inteligente de recursos ou que de alguma maneira influ�nciavam a experi�ncia ou expectativa do usu�rio.
Diversas trabalhos em torno de cheiros de c�digo vem sendo realizadas ao longo dos �ltimos anos. J� existem inclusive diversos cheiros de c�digo mapeados, por�m poucos deles s�o espec�ficos da plataforma Android \cite{Mannan_Dig_Ahmed_Jensen_Abdullah_Almurshed}. Segundo Hecht \cite{Hecht2015} estudos sobre cheiros de c�digo de c�digo sobre aplica��es Android ainda est�o em sua inf�ncia. Outro ponto que reafirma esta quest�o s�o os trabalhos de Linares-V�squez \cite{DomainMatters} e Hecht \cite{Hecht2015} onde concluem que, em projetos Android, � mais comum cheiros de c�digo espec�ficos do que cheiros de c�digo Orientado a Objetos. 

O trabalho de Verloop \cite{MobileSmells:13} avalia a presen�a de cheiros de c�digo definidos por Fowler \cite{Refactoring:99} e Minelli e Lanza \cite{Mantyla2013} em projetos Android. Apesar das relevantes contribui��es feitas, a conclus�o sobre a incid�ncia de tais cheiros de c�digo n�o � plenamente conclusiva, visto que dos 6 cheiros de c�digo analisados (\textit{Large Class}, \textit{Long Method}, \textit{Long Parameter List}, \textit{Type Checking}, \textit{Feature Envy} e \textit{Dead Code}) apenas dois deles, \textit{Long Method} e \textit{Type Checking}, se apresentam com maior destaque (duas vezes mais prov�vel) em projetos Android. Os demais apresentam uma diferen�a m�nima em classes Android quando se comparados a classes n�o espec�ficas do Android. Por fim, acaba por n�o ser conclusivo quanto a maior relev�ncia deles em Android ou n�o. 

Desta forma, Verloop \cite{MobileSmells:13} conclui com algumas recomenda��es de refatora��o de forma a mitigar a presen�a do mau cheiro \textit{Long Method}. Estas recomenda��es s�o o uso do atualmente j� reconhecido padr�o \textit{ViewHolder} em classes do tipo \texttt{Adapters}. Ele tamb�m sugere um \textit{ActivityViewHolder} de forma a extrair c�digo do m�todo \texttt{onCreate} e deix�-lo menor. Sugere tamb�m o uso do atributo \texttt{onClick} em XMLs de \textsc{layout} e \textsc{menu}. 

% Sobre esta �ltima recomenda��o, acredito que ser� entendida como um anti-pattern, visto que acopla o comportamento a um layout espec�fico al�m do que hoje j� � poss�vel o uso de lambdas tamb�m no desenvolvimento android, encurtando muito portanto o tamanha das antigas classes an�nimas e o uso de resource ids ao inv�s dos resources values, onde o c�digo fica mais verboso, logo, mais complicado de ler, ele recomenda sempre que poss�vel, usar o resource id ao inv�s do resource value. Apesar disso, sabemos que c�digo leg�vel resulta em tempo de desenvolvimento menor, pois c�digos complexos dificultam o desenvolvimento de software. Das ferramentas de an�lise de c�digo analisadas por Verloop, a �nica que suporta XML � a Lint, desenvolvida pela Google para Android. [DIFEREN�A] 

Diferentemente de validar a presen�a de cheiros de c�digo previamente catalogados conforme feito por Verloop \cite{MobileSmells:13}, esta disserta��o objetiva identificar, catalogar e validar, com base na experi�ncia de desenvolvedores, boas e m�s pr�ticas espec�ficas � \textbf{camada de apresenta��o de projetos Android}. 

Outro trabalho muito relevante realizado nesse tema � o de Reimann et al. \cite{ReimannBrylski2013} que, baseado na documenta��o do Android e sites t�cnicos, documentou 30 \textit{quality smells} espec�ficos para Android. No texto, \textit{quality smells} s�o definidos como \textit{``uma determinada estrutura em um modelo, indicando que influ�ncia negativamente requisitos espec�ficos de qualidade, que podem ser resolvidos por refatora��es particulares ao modelo''} (tradu��o livre). Entretanto, esses requisitos de qualidade s�o centrados no usu�rio (estabilidade, tempo de in�cio, conformidade com usu�rio, experi�ncia do usu�rio e acessibilidade), consumo inteligente de recursos (efici�ncia geral e no uso de energia e mem�ria) e seguran�a. Aspectos relacionados a qualidade de c�digo n�o s�o considerados nesse trabalho de Reimann et al. \cite{ReimannBrylski2013}. Esta disserta��o se difere do trabalho Reimann et al. \cite{ReimannBrylski2013} pois pretende-se encontrar cheiros de c�digo em termos de qualidade de software, ou seja, que influenciam na legibilidade e manutenibilidade do c�digo do projeto.

% % -*- root: dissertacao.tex -*-
%%%%%%%%%%%%%%%%%%%%%%%%%%%%%%%%%%%%%%%%%%%%%%%%%%%%%%%%%%%%%%%%%%%%%%%
\setlength{\parindent}{20pt}
\setlength{\textheight}{22cm}
\setlength{\parskip}{0.2cm}
\linespread{1.2} % Para aumentar o espa�amento entre as linhas
%%%%%%%%%%%%%%%%%%%%%%%%%%%%%%%%%%%%%%%%%%%%%%%%%%%%%%%%%%%%%%%%%%%%%%%

\chapter{Camada de Apresenta��o Android}
\label{ch:PresentationLayer}

Um assunto essencial para o entendimento deste trabalho � explanar o que queremos dizer com ``Camada de Apresenta��o Android''. Nesta se��o abordamos justamente este assunto de forma a explanar como chegamos na defini��o aqui usada.

Em nossas pesquisas bibliogr�ficas n�o foi econtrada uma defini��o formal sobre camada de apresenta��o Android. Encontramos por�m, pontos na documenta��o oficial do Android \cite{AndroidDeveloperSite2016} que afirmam que determinado elemento de alguma forma � parte desta camada. Por exemplo o trecho sobre \textit{Activities} diz que ``representa uma tela com interface do usu�rio''. O trecho sobre recursos do aplicativo afirma que ``um aplicativo Android � composto por outros arquivos al�m de c�digo Java, ele requer recursos como imagens, arquivos de �udio e qualquer recurso relativo a apresenta��o visual do aplicativo'' \cite{AndroidFundamentals}. Encontramos tamb�m postagens em sites t�cnicos sobre Android que de alguma forma indicam que determinado elemento comp�e a camada de apresenta��o Android, por exemplo Preussler relaciona \textit{adapters} como parte da camada de apresenta��o \cite{AdaptersPreussler2016}. Desta forma viu-se necess�rio definir quais s�o os elementos, para efeitos desta disserta��o, que comp�em a camada de apresenta��o em aplicativos Android. 

Os prim�rdios de GUI (\textit{Graphical User Interfaces} ou Interfaces de Usu�rio Gr�ficas) foram em 1973 com o projeto Alto, desenvolvido pelos pesquisadores da Xerox Palo Alto Research Center (PARC), seguido do projeto Lisa da Apple em 1979. Estes dois projetos serviram de base e inspira��o para o Machintosh, lan�ado pela Apple em 1985. As primeiras defini��es sobre GUI que surgiram nessa �poca abordavam sobre componentes de uso comum como �cones, janelas, barras de rolagem, menus supensos, bot�es, caixas de entrada de texto; gerenciadores de janelas; arquivos de �udio, internacionaliza��o e eventos. Antes deste per�odo existiam apenas interfaces de linha de comando \cite{GUIRaymond2004, UITecMundo2016}.

Outra fonte define camada de apresenta��o como ``informa��es gr�ficas, textuais e auditivas apresentadas ao utilizador, e as sequ�ncias de controle (como comandos de teclado, \textit{mouse} ou toque) para interagir com o programa'' \cite{UIWikipedia2016}. 

Unindo as defini��es supracitadas, definimos que todos os elementos do Android que s�o apresentados ou interagem com o usu�rio de alguma forma auditiva, visual ou por comando de voz ou toque s�o elementos da \textbf{Camada de Apresenta��o}, s�o eles:

\begin{itemize}
	\item \textbf{Activities e Fragments} Representam uma tela ou um fragmento de tela. A exemplo temos classes Java que herdam de \texttt{Activity}, \texttt{Fragment} ou classes similares.

	\item \textbf{Listeners} Meio pelo qual os comandos do usu�rio s�o capturados pelo aplicativo. A exemplo temos classes Java que implementam interfaces como \texttt{View.OnClickListener}.

	\item \textbf{Recursos do Aplicativo} Arquivos que apresentam textos, imagens, �udios, menus, interfaces gr�ficas (\textit{layout}), dentre outros. Est�o inclu�dos neste item todos os arquivos dentro do diret�rio \texttt{res} ainda que em seu formato Java. A exemplo podemos citar classes que herdam da classe \texttt{View} ou \texttt{ViewGroup}.

	\item \textbf{Adapters} Meio pelo qual s�o carregados conte�dos din�micos ou n�o pr�-determinados na tela. A exemplo podemos citar classes que herdam da classe \texttt{BaseAdapter}.

\end{itemize}


% -*- root: dissertacao.tex -*-
%%%%%%%%%%%%%%%%%%%%%%%%%%%%%%%%%%%%%%%%%%%%%%%%%%%%%%%%%%%%%%%%%%%%%%%
\setlength{\parindent}{20pt}
\setlength{\textheight}{22cm}
\setlength{\parskip}{0.2cm}
\linespread{1.2} % Para aumentar o espa�amento entre as linhas
%%%%%%%%%%%%%%%%%%%%%%%%%%%%%%%%%%%%%%%%%%%%%%%%%%%%%%%%%%%%%%%%%%%%%%%

\chapter{Proposta de Disserta��o}

Conforme apresentado no Cap�tulo 1, podem existir cheiros de c�digo espec�ficos a um dom�nio, tecnologia ou plataforma (por exemplo, Android) \cite{FinavaroAniche2016, DomainMatters, MobileSmells:13}. Geoffrey \cite{Hecht2015} afirma que a detec��o e especifica��o de padr�es m�veis ainda � um problema em aberto e que \textit{antipatterns} Android s�o mais frequentes em projetos m�veis do que \textit{antipatterns} orientado a objetos. Pesquisas em torno de projetos de aplicativos m�veis ainda s�o poucas \cite{Mannan_Dig_Ahmed_Jensen_Abdullah_Almurshed}. Desta forma, neste cap�tulo � apresentada a proposta da disserta��o e o cronograma de atividades planejadas. 


\section{Atividades}

Para obter as ideias iniciais para a deriva��o dos cheiros de c�digo na camada de apresenta��o Android, foi aplicado um question�rio sobre boas e m�s pr�ticas Android na comunidade de desenvolvedores do Brasil e exterior. O question�rio pode ser encontrado no Ap�ndice A e at� o momento da escrita desta proposta de qualifica��o foram coletadas 44 respostas. Ainda de modo a complementar os dados coletados com o question�rio, pretende-se realizar entrevista com desenvolvedores Android sobre o mesmo tema. Tamb�m ser� feito uma an�lise para derivar os cheiros de c�digo, essa an�lise ser� feita com base em estrat�gias j� utilizadas em trabalhos anteriores como o de Aniche et al. \cite{FinavaroAniche2016}. Para reduzir vi�s sobre os cheiros de c�digo definidos, os mesmos ser�o validados com mais de um especialista em Android. A deriva��o dos cheiros de c�digo est� relacionada a Q1 definida na se��o 1.1 e as atividades planejadas s�o:

\begin{itemize} 
	\item Bibliografia e Trabalhos Relacionados.
	\item Survey Boas e M�s pr�ticas Android.
	\item Deriva��o dos cheiros de c�digo.
	\item Valida��o cheiros de c�digo com Especialista.
\end{itemize}

Evid�ncias na literatura sugerem que cheiros de c�digo de c�digo podem esconder manutenibilidade de c�digo \cite{Sjoberg_Quantifying_2013, Yamashita6405287, Yamashita:2013:EII:2486788.2486878} e aumentar a tend�ncia a mudan�as e introdu��o de defeitos \cite{Khomh:2009:ESI:1685994.1686210, Khomh:2012:ESI:2158916.2158921}. Mario et al. \cite{DomainMatters} mostra que \textit{antipatterns} impactam negativamente m�tricas relacionadas a qualidade em projetos m�veis, em particular m�tricas relacionadas a propens�o de falhas. Desta forma, pretende-se avaliar o impacto dos cheiros de c�digo propostos na tend�ncia a mudan�as e introdu��o de defeitos no c�digo. Para isso ser� realizado um experimento presencial com desenvolvedores Android. Este experimento est� relacionado ao Q2 definida na Se��o 1.1 e a atividade planejada �:

\begin{itemize} 
	\item Experimento Impacto em Mudan�as/Defeitos.
\end{itemize}

Evid�ncias na literatura tamb�m sugerem que cheiros de c�digo de c�digo s�o percebidos por desenvolvedores \cite{Palomba_Do_2014}, desta forma pretende-se avaliar se desenvolvedores Android percebem c�digos afetados pelos cheiros de c�digo propostos como indicativos de trechos de c�digos ruins. Para isso ser� conduzido outro experimento tamb�m com desenvolvedores Android. Esse experimento est� relacionado a Q3 definida na Se��o 1.1 e a seguinte atividade est� planejada:

\begin{itemize} 
	\item Experimento Percep��o Desenvolvedores.
\end{itemize}


\section{Cronograma}

Na Tabela \ref{tab:Cronograma} s�o apresentadas as atividades previstas para a conclus�o da disserta��o bem como em qual per�odo pretende-se realiz�-la.

\begin{table}[h]
\centering

\newcommand\T{\rule{0pt}{2.6ex}}       % Top strut
\newcommand\B{\rule[-1.2ex]{0pt}{0pt}} % Bottom strut

\begin{tabular}{|l|cc|cccc|} 
\hline
\multicolumn{1}{|c|}{} 	& \multicolumn{2}{c|}{2016}   	& \multicolumn{4}{c|}{2017} \\
\textbf{Atividades}		& 3$^o$ Tri & 4$^o$ Tri 		& 1$^o$ Tri & 2$^o$ Tri & 3$^o$ Tri & 4$^o$ Tri \\
\hline
\hline
% \rule{0pt}{-1em} 		& 			& 					& 			& 			&			&			\\
Bibliografia e Trabalhos Relacionados 	& \textbullet 	& \textbullet	& 				& 				& 				& 				\T \\
Survey Boas e M�s pr�ticas Android 		& 				& \textbullet	& 				& 				& 				& 				\\
Entrevista Boas e M�s pr�ticas Android	& 				& 				& \textbullet	& 				& 				& 				\\
Deriva��o dos cheiros de c�digo 				& 				& 				& \textbullet	& 				& 				& 				\\
Valida��o cheiros de c�digo c/ Especialista	& 				& 				& \textbullet	& 				& 				& 				\\
Experimento Impacto em Mudan�as/Defeitos	& 				& 				& 				& \textbullet	& 				& 				\\
Experimento Percep��o Desenvolvedores	& 				& 				& 				& \textbullet 	& 				& 				\\
Escrita da Disserta��o					& \textbullet	& \textbullet	& \textbullet	& \textbullet	& \textbullet	& \textbullet	\\
Defesa									& 				& 				& 				& 				& 				& \textbullet	\B \\
\hline
\end{tabular}
\caption{Cronograma de atividades propostas.}
\label{tab:Cronograma}
\end{table} 
% % -*- root: dissertacao.tex -*-
%%%%%%%%%%%%%%%%%%%%%%%%%%%%%%%%%%%%%%%%%%%%%%%%%%%%%%%%%%%%%%%%%%%%%%%
\setlength{\parindent}{20pt}
\setlength{\textheight}{22cm}
\setlength{\parskip}{0.2cm}
\linespread{1.2} % Para aumentar o espa�amento entre as linhas
%%%%%%%%%%%%%%%%%%%%%%%%%%%%%%%%%%%%%%%%%%%%%%%%%%%%%%%%%%%%%%%%%%%%%%%

\chapter{Pesquisa}
\label{ch:research}

\section{Camada de Apresenta��o Android}
\label{sc:PresentationLayer}


\section{Defini��o dos Maus Cheiros}
\label{sc:CodeSmellsDefinition}

\subsection{Coleta de Dados}


\subsection{An�lise dos Dados}


\subsection{Valida��o com Especialistas}


\section{Percep��o dos Desenvolvedores}
\label{sc:DevPerceptions}
 
% % -*- root: dissertation.tex -*-
Nesta se��o n�s revisitamos as quest�es de pesquisa e sugerimos trabalhos futuros.

\subsection{Quest�es de Pesquisa Revisitadas}

Boas pr�ticas e cheiros de c�digo s�o um grande recurso para desenvolvedores aumentarem a qualidade e a manutenibilidade de seus sistemas. No entanto, a maioria dos cat�logos de maus cheiros mais populares surgiram em meados da d�cada de 70, quando muitas tecnologias ainda n�o existiam, inclusive o Android, foco desta pesquisa. Neste estudo, definimos e validamos empiricamente um cat�logo com 20 maus cheiros espec�ficos � camada de apresenta��o Android. Ou seja, podem afetar c�digos como: \textit{Activities}, \textit{Fragments}, \textit{Adapters}, \textit{Listeners}, recursos de \textit{Layout}, \textit{Style}, \textit{String} ou \textit{Drawables}. Fizemos isso a partir de dois question�rios online e um experimento de c�digo tamb�m online. 

No primeiro question�rios, respondido por 45 desenvolvedores Android, mapeamos 20 maus cheiros na camada de apresenta��o Android. No segundo question�rio, respondido por 201 desenvolvedores, validamos que desenvolvedores Android consideram os maus cheiros propostos importantes de serem mitigados e frequentes no dia a dia de desenvolvimento. Por �ltimos, com o experimento online, respondido por 70 desenvolvedores, validamos que desenvolvedores Android consideram os c�digos afetados pelos maus cheiros propostos como c�digos problem�ticos. Ao todo participaram da pesquisa 316 desenvolvedores Android. 

A seguir revisitamos cada quest�o de pesquisa e suas respostas: \\

% Notamos que, maus cheiros relacionados a componentes Android s�o considerados problemas mais severos do que os maus cheiros relacionados a recursos do aplicativo. Aprendemos que, al�m dos maus cheiros tradicionais, os maus cheiros espec�ficos tamb�m podem ser problem�ticos para a manuten��o de aplicativos Android. 


% Desenvolvedores de aplicativos Android j� podem come�ar a se beneficiar do nosso cat�logo. 

% Resumo
% Quest�es de pesquisa
% Trabalhos futuros pode ser uma frase
% Discuss�es
%   - pq ser� que os maus cheiros em recursos s�o mais frequentes
%   - outras no arquivo de discussoes
% Frase positiva


% Neste artigo investigamos a exist�ncia de boas e m�s pr�ticas no \textit{front-end} de projetos Android. Fizemos isso atrav�s de um estudo explorat�rio qualitativo com 45 desenvolvedores, onde mapeamos 23 m�s pr�ticas Android. Ap�s, validamos a percep��o de desenvolvedores Android sobre as quatro m�s pr�ticas mais recorr�ntes. Fizemos isso atrav�s de um experimento online respondido com 20 desenvolvedores Android. Respondemos a \textbf{QP$_1$} com um cat�logo com 23 m�s pr�ticas no \textit{front-end} Android. Respondemos a \textbf{QP$_2$} atrav�s da valida��o com sucesso da percep��o de desenvolvedores sobre 2 das m�s pr�ticas de alta recorr�ncia.

% Neste artigo investigamos a exist�ncia de boas e m�s pr�ticas no \textit{front-end} de projetos Android. Fizemos isso atrav�s de um estudo explorat�rio qualitativo com 45 desenvolvedores, onde mapeamos 23 m�s pr�ticas Android. Ap�s, validamos a percep��o de desenvolvedores Android sobre as quatro m�s pr�ticas mais recorr�ntes. Fizemos isso atrav�s de um experimento online respondido com 20 desenvolvedores Android. 

% Respondemos a \textbf{QP$_1$} com um cat�logo com 23 m�s pr�ticas no \textit{front-end} Android. Respondemos a \textbf{QP$_2$} atrav�s da valida��o com sucesso da percep��o de desenvolvedores sobre 2 das m�s pr�ticas de alta recorr�ncia.

% \begin{center}
  \textbf{QP$_1$ Existem maus cheiros que s�o espec�ficos a camada de apresenta��o Android?}
% \end{center}

Certamente existem diversas formas de se implementar c�digos em elementos da camada de apresenta��o Android. Algumas destas formas s�o consideradas melhores e outras piores por desenvolvedores Android. Partindo dessa percep��o, propomos 20 maus cheiros de c�digo espec�ficos a elementos da camada de apresenta��o Android. Estes elementos compreendem componentes Java derivados do Android SDK como: \textit{Activities}, \textit{Fragments}, \textit{Listeners} e \textit{Adapters} e os principais recursos do aplicativo: \textit{Layout}, \textit{Styles}, \textit{String} e \textit{Drawable}. \\


% \begin{center}
  \textbf{QP$_2$ Com qual frequ�ncia os maus cheiros s�o percebidos e o qu�o importante s�o considerados pelos desenvolvedores?}
% \end{center}

Todos os maus cheiros propostos foram considerados com algum n�vel de frequ�ncia no dia a dia de desenvolvimento, alguns mais frequentes que outros. Dentre os elementos da camada de apresenta��o, notamos que os desenvolvedores percebem mais frequentemente a presen�a de maus cheiros relacionados a recursos do que aos componentes da camada de apresenta��o Android. A percep��o de import�ncia em mitig�-los � alta em todos os maus cheiros. \\


% \begin{center}
 \textbf{QP$_3$ Desenvolvedores Android percebem os c�digos afetados pelos maus cheiros como problem�ticos?}
% \end{center}

Validamos a percep��o de desenvolvedores sobre 7 dos 20 maus cheiros propostos, s�o ele: \textsc{\small Componente de UI C�rebro}, \textsc{\small Componente de UI Acoplado}, \textsc{\small Adapter Complexo}, \textsc{\small Longo Recurso de Estilo}, \textsc{\small Comportamento Suspeito}, \textsc{\small Layout Profundamente Aninhado} e \textsc{\small Atributos de Estilo Repetidos}. Nossos resultados mostram que desenvolvedores percebem os c�digos afetados pelos maus cheiros como problem�ticos. Foi poss�vel validar estatisticamente a percep��o em 6 maus cheiros. Sobre o mau cheiro \textsc{\small Longo Recurso de Estilo}, n�o obtivemos dados o suficiente para afirmar ou n�o a percep��o de desenvolvedores sobre ele.\\

\subsection{Trabalhos Futuros}

Acreditamos que nossas contribui��es representam um pequeno, por�m importante passo, na busca por mais qualidade de c�digo na plataforma Android. Pesquisadores e desenvolvedores Android j� podem se beneficiar dos nossos resultados. Pesquisadores podem utilizar nossos resultados como ponto de partida para a defini��o de heur�sticas de identifica��o dos maus cheiros, propor refatora��es vi�veis, validar os outros maus cheiros aqui n�o validados, revalidar os maus cheiros validados a partir de c�digos diferentes ou mesmo implementar ferramentas que os identifiquem de automaticamente. 

Desenvolvedores Android j� podem utilizar nossos cat�logo de maus cheiros Android como aux�lio na identifica��o de c�digos problem�ticos para serem melhorados, ainda que de forma manual.

% Notamos que muitas vezes as respostas�para as quest�es sobre sobre boas pr�ticas apresentada no 1o question�rio, sobre boas e m�s pr�ticas em elementos Android, vieram na forma de sugest�es de como solucionar o que o participante indicou como m� pr�tica para aquele elemento. Como n�o foi o foco desta pesquisa validar se a sugest�es dadas como solu��es ao mau cheiro de fato se aplica, n�o exploramos a fundo estas informa��es. Entretando, disponibilizamos uma tabela que indica a boa pr�tica sugerida para cada mau cheiro definido no ap�ndice \ref{appendix:smells-purpose-of-solution}.
 


\appendix

%%%%%%%%%%%%%%%%%%%%%%%%%%%%%%%%%%%%%%%%%%%%%%%%%%%%%%%%%%%%%%%%%%%%%%%
\setlength{\parindent}{0pt}
\setlength{\textheight}{22cm}
\setlength{\parskip}{0.2cm}

% Para aumentar o espa�amento entre as linhas
\linespread{1.2}
%%%%%%%%%%%%%%%%%%%%%%%%%%%%%%%%%%%%%%%%%%%%%%%%%%%%%%%%%%%%%%%%%%%%%%%

\chapter{Question�rio sobre Boas e M�s Pr�ticas}

\textbf{Boas e m�s pr�ticas na Camada de Apresenta��o de aplicativos Android nativos.}

Esta pesquisa � focada em boas e m�s pr�ticas na camada de apresenta��o de aplicativos Android nativos. Nosso objetivo � melhorar a qualidade no desenvolvimento de aplicativos Android. Ajude-nos a identificar quais s�o estas boas e m�s pr�ticas. Este question�rio deve durar em torno de 10 a 15 minutos e compreendo que aparentemente parece muito tempo, mas perceber� que � pouco para um material que poder� lhe ser �til num futuro. Obrigada! 

\textbf{Um pouco sobre voc�:}
\begin{enumerate}
	\item Qual sua posi��o profissional atual?
	\item Anos de experi�ncia com desenvolvimento de software? (1-10+)
	\item Anos de experi�ncia com desenvolvimento de software com Java? (1-10+)
	\item Anos de experi�ncia com desenvolvimento de aplicativos Android nativo? (1-10+)
	\item Quantos aplicativos Android que foram para produ��o voc� colaborou? 
	\item Com qual frequ�ncia voc� estuda/l� sobre boas pr�ticas de desenvolvimento de aplciativos Android nativo? (1-5)
\end{enumerate}

\textbf{Sobre as boas e m�s pr�ticas.}
\begin{enumerate}

	\item \textbf{Activities}
	\begin{itemize}
		\item Quais **BOAS** pr�ticas voc� costuma utilizar ao lidar com Activities?
		\item O que voc� considera **M�S** pr�ticas ao lidar com Activities?
	\end{itemize}

	\item \textbf{Fragments}
	\begin{itemize}
		\item Quais **BOAS** pr�ticas voc� costuma utilizar ao lidar com Fragments?
		\item O que voc� considera **M�S** pr�ticas ao lidar com Fragments?
	\end{itemize}

	\item \textbf{Adapters}
	\begin{itemize}
		\item Quais **BOAS** pr�ticas voc� costuma utilizar ao lidar com Adapters?
		\item O que voc� considera **M�S** pr�ticas ao lidar com Adapters?
	\end{itemize}

	\item \textbf{Listeners}
	\begin{itemize}
		\item Quais **BOAS** pr�ticas voc� costuma utilizar ao lidar com Listeners?
		\item O que voc� considera **M�S** pr�ticas ao lidar com Listeners?
	\end{itemize}

	\item \textbf{Recursos de Layouts}
	\begin{itemize}
		\item Quais **BOAS** pr�ticas voc� costuma utilizar ao lidar com recursos de Layouts?
		\item O que voc� considera **M�S** pr�ticas ao lidar com recursos de Layouts?
	\end{itemize}

	\item \textbf{Recursos de Menu}
	\begin{itemize}
		\item Quais **BOAS** pr�ticas voc� costuma utilizar ao lidar com recursos de Menus?
		\item O que voc� considera **M�S** pr�ticas ao lidar com recursos de Menus?
	\end{itemize}

	\item \textbf{Recursos Gr�ficos}
	\begin{itemize}
		\item Quais **BOAS** pr�ticas voc� costuma utilizar ao lidar com recursos Gr�ficos?
		\item O que voc� considera **M�S** pr�ticas ao lidar com recursos Gr�ficos?
	\end{itemize}

	\item \textbf{Recursos de Anima��o}
	\begin{itemize}
		\item Quais **BOAS** pr�ticas voc� costuma utilizar ao lidar com recursos de Anima��o?
		\item O que voc� considera **M�S** pr�ticas ao lidar com recursos de Anima��o?
	\end{itemize}

	\item \textbf{Recursos de Estilos}
	\begin{itemize}
		\item Quais **BOAS** pr�ticas voc� costuma utilizar ao lidar com recursos de Estilos?
		\item O que voc� considera **M�S** pr�ticas ao lidar com recursos de Estilos?
	\end{itemize}

	\item \textbf{Recursos de Cor}
	\begin{itemize}
		\item Quais **BOAS** pr�ticas voc� costuma utilizar ao lidar com recursos de Cor?
		\item O que voc� considera **M�S** pr�ticas ao lidar com recursos de Cor?
	\end{itemize}

	\item \textbf{Recursos de Textos}
	\begin{itemize}
		\item Quais **BOAS** pr�ticas voc� costuma utilizar ao lidar com recursos de Textos?
		\item O que voc� considera **M�S** pr�ticas ao lidar com recursos de Textos?
	\end{itemize}

	\item \textbf{Recursos de Inteiros, Booleanos, Dimens�es ou Arrays}
	\begin{itemize}
		\item Quais **BOAS** pr�ticas voc� costuma utilizar ao lidar com recursos de Inteiros, Booleanos, Dimens�es ou Arrays?
		\item O que voc� considera **M�S** pr�ticas ao lidar com recursos de Inteiros, Booleanos, Dimens�es ou Arrays?
	\end{itemize}
\end{enumerate}

\textbf{�ltimos Pensamentos}
\begin{enumerate}
	\item Existe alguma outra *BOA* pr�tica na camada de apresenta��o que n�s n�o perguntamos a voc� ou que voc� ainda n�o mencionou?
	\item Existe alguma outra *M�* pr�tica na camada de apresenta��o que n�s n�o perguntamos a voc� ou que voc� ainda n�o mencionou?
	\item Podemos entrar em contato para outras etapas desta pesquisa se for necess�rio?
	\item Deixe seu email para participar de futuras pesquisas necess�rias ainda para a conclus�o do trabalho ou mesmo para receber os resultados desta pesquisa. 
\end{enumerate}






%%%%%%%%%%%%%%%%%%%%%%%%%%%%%%%%%%%%%%%%%%%%%%%%%%%%%%%%%%%%%%%%%%%%%%%%%

\onehalfspacing
\bibliographystyle{plain}
\bibliography{bibliografias}

\printindex

\end{document}
