


  ISO/IEC 9126 | substituido pela ISO/IEC 25010:2011 ou SQuaRE divide qualidade de software em 6 áreas () e sub-áreas.  
  ISO/IEC 25010:2011 (SQuaRE) | 3 perspectivas de qualidade  (interna, externa e em uso)   
  CISQ    | Divide qualidade de software em 5 conceitos () e foi fundado em 2011 e se apoia nas definições do ISO 9126.
  SWEBOK  | Divite qualidade de software em 4 áreas e considera que qualidade está envolvido com as qualidades estáticas do software.
  FURPS 


\begin{table}[h]
\centering
\renewcommand*{\arraystretch}{1}
\small
\begin{tabular}{@{}l|c@{}}
\toprule
\textbf{Anos de Experiência} & \textbf{Participantes} \\
\hline
ISO/IEC 9126                  & Substituído pela ISO/IEC 25010:2011 (SQuaRE) decompõe qualidade de software em 6 áreas () e sub-áreas.   \\
ISO/IEC 25010:2011 (SQuaRE)   & 3 perspectivas de qualidade (interna, externa e em uso)    \\
CISQ                          & Decompõe qualidade de software em 5 conceitos () e foi fundado em 2011 e se apoia nas definições do ISO 9126. \\
SWEBOK                        & Decompõe qualidade de software em 4 áreas e considera que qualidade está envolvido com as qualidades estáticas do software. \\
FURPS                         & Decompõe qualidade de software em funcionalidade, usabilidade, confiabilidade, desempenho e suporte. \\
\toprule
\end{tabular}
\caption{Experiência dos participantes com desenvolvimento Android.}
\label{tab:DadosDemograficos}
\end{table}