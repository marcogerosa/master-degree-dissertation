% -*- root: dissertation.tex -*-
\section{Discuss�es}

Notamos que muitas vezes as respostas�para as quest�es sobre sobre boas pr�ticas apresentada no 1o question�rio, sobre boas e m�s pr�ticas em elementos Android, vieram na forma de sugest�es de como solucionar o que o participante indicou como m� pr�tica para aquele elemento. Como n�o foi o foco desta pesquisa validar se a sugest�es dadas como solu��eo ao mau cheiro de fato se aplica, n�o exploramos a fundo estas informa��es. Entretando, disponibilizamos uma tabela que indica a boa pr�tica sugerida para cada mau cheiro definido no ap�ndice \ref{appendix:smells-purpose-of-solution}.

% \todo{Como os achados do estudo ajudam a melhorar o desenvolvimento de aplica��es para Android, ou apoiar a identifica��o, prioriza��o e minimiza��o de m�s pr�ticas?}


% \todo{Relacionar as m�s pr�ticas identificadas com os maus cheiros tradicionais. O que h� de semelhan�a, faz sentido definir um diferente/espec�fico ao android ou basta aplicar o conceito do tradicional a plataforma (exemplo de style grande pode ser aplicado o conceito de large class?)?}


% - apesar de atingirmos pessoas de outros pa�ses, mais de 80\% s�o de brasileiros
% - esbarramos em resultados iguais de outras pesquisas sobre q activities e adapter s�o afetadas por god class
% - muitas boas pr�ticas vieram no formato de sugest�o de como resolver a m� pr�tica

