% -*- root: index.tex -*-
\section{Boas Pr�ticas, Padr�es, Anti-Padr�es e Maus Cheiros}

� comum no dia-a-dia de desenvolvimento de software ouvirmos o termos \textit{boas pr�ticas} como uma generaliza��o para se referir a \textit{padr�es de projetos}, \textit{anti-padr�es} e \textit{maus cheiros de c�digo}. Apesar dessa comum generaliza��o, cada um destes termos possuem significados levemente distintos que os diferenciam entre si. \\

Dessa forma, esta sess�o tem como objetivo fundamentar e definir esses termos, para efeitos deste trabalho, objetivando tornar mais claro o entendimento da pesquisa.


\subsection{Boas Pr�ticas}

Segundo o dicion�rio Aur�lio [1], \textit{pr�tica} significa ``maneira habitual de proceder''. Dessa forma, podemos dizer que \textit{boas pr�ticas} � ``um conjunto de maneiras habituais de proceder consideradas boas, e portanto recomendadas, sobre uma determinada �rea''.

Uma \textit{melhor pr�tica} � segundo [2], ``um m�todo ou t�cnica geralmente aceito como superior a qualquer outra alternativa porque produz resultados superiores'' ou ainda, pode significar ``uma forma de se fazer algo que foi aceita como padr�o''. \\

A leve diferen�a entre um termo e outro tem sido discutida em diversos trabalhos [2, 3] mas n�o altera seu significado. Para efeitos deste texto, usaremos o termo \textbf{boas pr�ticas} para representar o \textbf{conjunto de pr�ticas consideradas boas, e portanto recomendadas, por desenvolvedores no desenvolvimento de aplicativos Android}.


\subsection{Padr�es de Projeto}




\subsection{Anti-Padr�es}

\subsection{Maus Cheiros}




% BIO
% =================================================================

% [1] https://dicionariodoaurelio.com/pratica
% [2] https://en.wikipedia.org/wiki/Best_practice #Critique
% [3] http://www.farrell-associates.com.au/Papers/Best%20Practice.pdf