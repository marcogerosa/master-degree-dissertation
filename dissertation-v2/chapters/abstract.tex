% -*- root: dissertation.tex -*-
\noindent CARVALHO, G. S. \textbf{Bad smells on the Android front-end: A study on the developers' perception}. 
2018. %100 f.
Disserta��o (Mestrado) - Instituto de Matem�tica e Estat�stica,
Universidade de S�o Paulo, S�o Paulo, 2018.
\\

There is no question that good codes matter, but how do you know when a code is not good? Bad smells of code help us identify problematic code snippets, but most of the bad odors cataloged are based on traditional technologies, created from the 1970s through the 90s, such as Java. There are still doubts about bad smells in technologies that have emerged in the last decade, such as Android, the main mobile platform in 2017 with more than 86\% market share. Some researchers have defined new bad smells related to Android eficience and usability. Other research concludes that the components most affected by traditional bad smells are related to the front-end, such as \textit{Activities} and \textit{Adapters}. Also noticed in some applications, front-end codes represent a larger part. It is noteworthy that the Android front-end is also composed of XML files, called application resources, used for a user interface (UI) construction, but these files were not considered in their analyzes. In this dissertation, we investigate existence of bad smells of code related to the Android front-end considering even application resources. We did this through 2 online surveys and 3 experiments summing 3XX developers. Our results showed that there is a common perception among practicing Android developers about bad practices no Android front-end. Our main contributions are a set of 13 bad smells from the Android front-end and a statistical analysis of the perceptions of practitioner developers about the bad smells set. Our contributions will serve researchers as a starting point for the definition of heuristics and implementation of automated tools and to practitioner developers as an aid in identifying problematic codes to be improved, even manually.


\noindent \textbf{Palavras-chave:} Android, bad smells, software quality, software maintance.