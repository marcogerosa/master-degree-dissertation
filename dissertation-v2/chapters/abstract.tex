% -*- root: dissertation.tex -*-
\noindent CARVALHO, G. S. \textbf{Anomalies in Android Presentation Layer: A Study On The Developers' Perception}. 
2018. %100 f.
Disserta��o (Mestrado) - Instituto de Matem�tica e Estat�stica,
Universidade de S�o Paulo, S�o Paulo, 2018.
\\

There is no question that good codes matter, but how do you know when code quality is low? Code smells help us identify problematic code snippets, but most of the code smells cataloged are based on traditional technologies, created from the 1970s through the 90s, such as Java. There are still doubts about code smells in technologies that have emerged in the last decade, such as Android, the main mobile platform in 2017 with more than 86\% market share. Some researchers have defined code smells related to Android efficiency and usability. Other research concludes that the components most affected by traditional code smells are related to the front-end, such as \textit{Activities} and \textit{Adapters}. Also noticed in some applications, front-end represent the larger part. It is noteworthy that the Android front-end is also composed of XML files, called application resources, used to build user interface (UI), but these files were not considered in their analyzes. In this dissertation, we investigate existence of code smells related to the Android front-end considering even application resources. To aim that performed two online surveys and three experiments summing 3XX developers. Our results show that there is a common perception among practicing Android developers about bad practices on Android front-end. Our main contributions are a catalog of 13 code smells about Android front-end and a statistical analysis of the perceptions of practicing developers about the code smells catalog. Our contributions will serve researchers as a starting point for the definition of heuristics and implementation of automated tools and to practitioners developers as an aid in identifying problematic codes to be improved, even manually.


\noindent \textbf{Palavras-chave:} software engineering, android, code smells, code quality, software maintenance, software anomalies.