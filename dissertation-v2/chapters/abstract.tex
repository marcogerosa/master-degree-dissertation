% -*- root: dissertation.tex -*-
\noindent CARVALHO, G. S. \textbf{Anomalies in the Presentation Layer of Android Applications}. 
2018. %100 f.
Disserta��o (Mestrado) - Instituto de Matem�tica e Estat�stica,
Universidade de S�o Paulo, S�o Paulo, 2018.
\\

We are aware that good code matters, but how to know when quality is low? Code smells, or anomalies, help us identify problematic code snippets, but most of the code smells cataloged are based on traditional practices and technologies, created from the 70s through the 90s, such as object oriented and Java. There are still doubts about code smells in technologies that have emerged in the last decade, such as Android, the main mobile platform in 2017 with more than 86\% market share. Some researchers have defined code smells related to Android efficiency and usability. Other research concludes that the components most affected by traditional code smells are related to the front-end components, such as \textit{Activities} and \textit{Adapters}. Also noticed in some applications, front-end code represent the larger part of the project's code. It is worth mentioning that the Android presentation layer is also composed of XML files, called resources, used to build the user interface (UI), but none of the cited research considered them in their analyzes. In this dissertation, we investigate the existence of code smells related to the Android front-end including application resources. We performed two online surveys and one online code experiment summing 316 developers. Our results show that there is a common perception among Android developers about bad practices on Android front-end. Our main contributions are a catalog of 20 code smells related to the Android front-end and a statistical analysis of the perceptions of developers about the 7 main code smells cataloged. Our contributions will provide to researchers a starting point for the definition of heuristics and implementation of automated tools and to developers as an aid in identifying problematic codes.


\noindent \textbf{Palavras-chave:} software engineering, android, code smells, code quality, software maintenance, software anomalies.
