� com grande prazer que enfim escrevo esta se��o da minha disserta��o! Certamente este manuscrito � fruto da colabora��o de diversas pessoas.

Primeiramente gostaria de agradecer a minha m�e! A pessoa mais importante pra mim e que com muito carinho me ensinou os valores fundamentais para superar desafios t�o intensos, como � cursar um mestrado. Obrigada m�e esteja voc� onde estiver! 

Agrade�o ao meu orientador Marco Aur�lio Gerosa por ter aceitado me acompanhar nessa trajet�ria, n�o ter desistido de mim nos momentos dif�ceis e por me orientar sempre de forma sincera e respeitosa. Muito obrigada!

Agrade�o ao meu amigo, colega de mestrado e conselheiro Maur�cio F. Aniche que com sua experi�ncia e empatia me guiou e ajudou de perto, em todos os momentos, antes e durante o mestrado. Certamente sua ajuda foi essencial e espero poder retribuir � altura!

Agrade�o aos meus colegas de mestrado e professores, em especial a minha amiga Ana Paula e ao Prof. Alfredo Goldman vel Lebman, que tamb�m foram muito especiais!

Agrade�o aos desenvolvedores que disponibilizaram alguns minutos do seu tempo para responder aos question�rios! Sem uma comunidade engajada e colaborativa esta pesquisa n�o teria acontecido! 

Certamente agrade�o a todos os pesquisadores anteriores, pelo qual pude obter o conhecimento necess�rio para partir de um ponto que n�o o zero. Espero, como eles, agregar valor na comunidade acad�mica de modo a tamb�m colaborar com outros pesquisadores que vierem a utilizar meus resultados.

Agrade�o a todos que participaram de alguma forma direta ou indiretamente e por ventura ou emo��o do momento eu esque�a de mencionar aqui. Obrigada! 

E finalmente, agrade�o ao meu parceiro David Robert que, mais uma vez, me ajudou e apoiou de todas as maneiras inimagin�veis durante essa (dif�cil) jornada. Obrigada por estar ao meu lado nos �ltimos anos e certamente estar� nos pr�ximos!

% Sempre muito compreens�vel e amoroso esteve ao meu lado e lutou comigo quando preciso! 



