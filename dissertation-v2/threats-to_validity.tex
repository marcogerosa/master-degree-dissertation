% -*- root: article.tex -*-
Uma limita��o deste artigo � que os dados foram coletados apenas a partir de question�rios online e o processo de codifica��o foi realizado apenas por um dos autores. Alterativas a esses cen�rios seriam realizar a coleta de dados de outras formas como entrevistas ou consulta a especialistas, e que o processo de codifica��o fosse feito por mais de um autor de forma a reduzir poss�veis enviezamentos.

Outra poss�vel amea�a � com rela��o a sele��o de c�digos limpos. Selecionar c�digos limpos � dif�cil. Sentimos uma dificuldade maior ao selecionar c�digos de recursos do Android. Um alternativa seria investigar a exist�ncia de ferramentas que fa�am esta selec�o, validar os c�digos selecionados com um especialista ou mesmo extender o teste piloto. 

Nossa pesquisa tenta replicar o m�todo utilizado por Aniche \cite{FinavaroAniche2016} ao investigar cheiros de c�digo no framework Spring MVC. Entretanto, nos deparamos com situa��es diferentes, das quais, ap�s a execu��o nos questionamos se aquele m�todo seria o mais adequado para todos os contextos neste artigo. Por exemplo, nosso resultado com a m� pr�tica RM nos levou a conjecturar se desenvolvedores consideram problemas em c�digos Java mais severos que problemas em recursos do aplicativo. O que nos levou a pensar sobre isso foi que, apesar do resultado, obtivemos muitas respostas que se aproximavam da deifni��o da m� pr�tica RM. Desta forma, levantamos que de todos os recursos avaliados, 74\% receberam severidade igual ou inferior a 3, contra apenas 30\% com os mesmo n�veis de severidade em c�digo Java. Desta forma, uma alternativa � repensar a forma de avaliar a percep��o dos desenvolvedores sobre m�s pr�ticas que afetem recursos do aplicativo.
