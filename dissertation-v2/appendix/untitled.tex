




\textsc{Layout Longo ou Repetido}          		& Indica como má prática código de layout} repetido ou muito grandes. De modo similar, respostas indicam como boas práticas extrair layout repetidos para reutilizá-los através da tag \textsc{include} ou extrair apenas com o objetivo de manter arquivos pequenos. Exemplos de frases que indicaram más práticas são: P41 diz ``copiar e colar layouts parecidos sem usar includes'', P23 diz ``[...] colocar muitos recursos no mesmo arquivo de layout.'', P8 diz ``considerar o ciclo de vida de fragments como os de activities''. Exemplos de frases que indicaram boas prática são: P32 diz ``Sempre quando posso, estou utilizando includes para algum pedaço de layout semelhante'', P36diz ``Criar layouts que possam ser reutilizados em diversas partes'' e P42 diz ``Separe um grande layout usando include ou merge''. \\ %&  \\

\textsc{Ausência de Diferentes Resoluções}  	& Indica como má prática ter apenas uma imagem para atender a todas as resoluções. De modo similar, respostas indicam como boas práticas ter a mesma imagem em diversos tamanhos para atender a resoluções diferentes. Exemplos de frases que indicaram más práticas são: P31 diz ``ter apenas uma imagem para multiplas densidades'', P4 diz ``Baixar uma imagem muito grande quando não é necessário. Há melhores formas de usar memória'', P44 diz ``Não criar [versões da] imagem para todos as resoluções''. Exemplos de frases que indicaram boas prática são: P34 diz ``Nada especial, apenas mantê-las em seus respectivos diretórios e ter variados tamanhos delas'', P36 diz ``Criar as pastas para diversas resoluções e colocar as imagens corretas''. O único elemento que entrou nessa categoria é o que representa imagens, \textsc{Drawable Resource}. \\ %&  \\

\textsc{Longo Recurso de Estilo}.       		& Indica como má prática o uso de apenas um arquivo para todos os \textsc{Styles Resources}. De modo similar, respostas indicam como boas práticas separar os estilos em mais de um arquivo. Exemplos de frases que indicaram más práticas são: P28 diz ``Deixar tudo no mesmo arquivo styles.xml'', P8 diz ``Arquivos de estilos grandes''. Exemplos de frases que indicaram boas prática são: P28 diz ``Se possível, separar mais além do arquivo styles.xml padrão, já que é possível declarar múltiplos arquivos XML de estilo para a mesma configuração''. P40 diz ``Divida-os. Temas e estilos é uma escolha racional''. O único elemento que entrou nessa categoria foi o \textsc{Style Resource}. \\


\textsc{Recurso de String Bagunçado}       		& Indica como má prática arquivos \textsc{String Resources}$^{3\star}$ desorganizados ou o uso de apenas um arquivo para todos os \textsc{String Resources}. De modo similar, respostas indicam como boas práticas separar as strings} em mais de um arquivo. Exemplos de frases que indicaram más práticas são: P28 diz ``Usar o mesmo arquivo strings.xml para tudo'', P42 diz ``Não orgaizar as strings quando o strings.xml começa a ficar grande''. Exemplos de frases que indicaram boas prática são: P28 diz ``Separar strings por tela em arquivos XML separados. Extremamente útil para identificar quais strings pertencentes a quais telas em projetos grandes''. P32 diz ``Sempre busco separar em blocos, cada bloco representa uma activity e nunca aproveito uma String pra outra tela''. O único elemento que entrou nessa categoria foi o \textsc{String Resource}. \\ %&  \\


\textsc{Atributos de Estilo Repetidos}     		& Indica como má prática a repetição de atributos de estilo nos \textsc{Layout Resource}. De modo similar, respostas indicam como boas práticas sempre que identificar atributos repetidos, extraí-los para um estilo. Exemplos de frases que indicaram más práticas são: P32 diz ``Utilizar muitas propriedades em um único componente. Se tiver que usar muitas, prefiro colocar no arquivo de styles.''. Exemplos de frases que indicaram boas prática são: P34 diz ``Sempre que eu noto que tenho mais de um recurso usando o mesmo estilo, eu tento movê-lo para o meu style resource}.''. Os elementos nessa categoria foram: \textsc{Layout Resources} e \textsc{Style Resources}. \\ %&  \\


\textsc{Reúso Excessivo de String}.       		& Indica como má prática reutilizar o mesmo \textsc{String Resource} em muitos lugares no aplicativo, apenas porque o texto coincide, pois caso seja necessário alterar em um lugar, todos os outros serão afetados. De modo similar, respostas indicam como boas práticas considerar a semântica ou contexto ao nomear um \textsc{String Resource}, para mesmo que o valor seja o mesmo, os recursos sejam diferentes. Exemplos de frases que indicaram más práticas são: P32 diz ``Utilizar uma String pra mais de uma activity, pois se em algum momento, surja a necessidade de trocar em uma, vai afetar outra.'', P6 diz ``Reutilizar a string em várias telas'' e P40 diz ``Reutilizar a string apenas porque o texto coincide, tenha cuidado com a semântica''. Exemplos de frases que indicaram boas prática são: P32 diz ``Sempre busco separar em blocos, cada bloco representa uma activity e nunca aproveito uma String pra outra tela.'' e P9 diz ``Não tenha medo de repetir strings [...]''. Apenas o elemento \textsc{String Resource} entrou nessa categoria. \\