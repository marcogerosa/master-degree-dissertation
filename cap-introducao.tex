%%%%%%%%%%%%%%%%%%%%%%%%%%%%%%%%%%%%%%%%%%%%%%%%%%%%%%%%%%%%%%%%%%%%%%%
\setlength{\parindent}{0pt}
\setlength{\textheight}{22cm}
\setlength{\parskip}{0.2cm}

% Para aumentar o espa�amento entre as linhas
\linespread{1.2}
%%%%%%%%%%%%%%%%%%%%%%%%%%%%%%%%%%%%%%%%%%%%%%%%%%%%%%%%%%%%%%%%%%%%%%%

\chapter{Introdu��o}

Smartphones existem desde 1993 por�m nesta �poca o foco era voltado para empresas e suas necessidades. Em 2007 houve o lan�amento do iPhone fabricado pela Apple, o primeiro smartphone voltado ao p�blico em geral. Ao final do mesmo ano o Google revelou seu sistema operacional para dispositivos m�veis, o Android \cite{SmartSociety:13}.

Nos anos que se seguiram diversas grandes empresas como Apple, Google, Blackberry e Microsoft entraram no mercado de smartphones e sistemas operacionais m�veis. Desde ent�o a ado��o do uso de smartphones e mais recentemente outros dispositivos m�veis conhecidos como dispositivos m�veis vest�veis (do ingl�s \textit{wearables}) tais como rel�gios inteligentes, pulseiras, dentre outros, tem crescido espantosamente.

Dentro do mercado de smartphone o Android tem ganhado for�a, o que resulta em mais pessoas usando dispositivos rodando aplica��es android. Desta forma, a demanda por aplicativos tem aumentado de forma acentuada. Logo, a quantidade de aplicativos desenvolvidos tem aumentado em propor��es similares. Isso significa que mais e mais projetos de aplicativos Android vem sendo constru�dos. 

Este crescimento de projetos de aplicativos na plataforma m�vel Android despertou a necessidade e interesse em explorar em mais profundidade quais s�o as particularidades de desenvolvimento nesta plataforma. Este trabalho vem contribuir com uma cole��o de \textit{code smells} validados, particulares a plataforma Android. 

\section{Motiva��o}\index{motiva��o}

A fazer. \\


\section{Objetivos}

A fazer. \\


\section{Abordagem de Solu��o}\index{abordagem}

A fazer. \\


\section{Originalidade e Relev�ncia}\index{originalidade}

A fazer. \\


\section{Organiza��o do Trabalho}\index{organiza��o}

A fazer. \\


